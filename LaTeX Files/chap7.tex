% Chapter 7
\chapter{قابلیت استفاده}
\section{تعریف قابلیت استفاده}
قابلیت استفاده\LTRfootnote{Usability}
به این تعریف می‌پردازد که چقدر برای کار انجام دادن یک وظیفه دلخواه راحت است. این ویژگی یکی از ارزان و راحت‌ترین ویژگی‌هایی است که می‌توان به وسیله آن کیفیت سیستم را افزایش داد. 
قابلیت استفاده شامل موارد زیر است:
\begin{itemize}
\item
ویژگی‌ها سیستم یادگیری
\item
استفاده موثر از یک سیستم 
\item
به حداقل رساندن تاثیر خطاها
\item
تطبیق سیستم با نیازهای کاربر
\item
افزایش اعتماد به نفس و رضایت
\end{itemize}


\section{سناریوی عمومی قابلیت استفاده}
سناریو‌ی عمومی قابلیت استفاده به صورت زیر است:
\begin{itemize}
\item
منبع محرک\LTRfootnote{Source of stimulus} : کاربر نهایی
\item
محرک\LTRfootnote{Stimulus} : کاربر نهایی که سعی می‌کند از سیستم به طور موثر استفاده کند و آن را یاد بگیرد و تاثیر خطاها را کمینه کند، با سیستم تطبیق پذیرد و آن را تنظیم کند.
\item
محصول\LTRfootnote{Artifact}: سیستم یا بخشی از سیستم که کاربر در حال تعامل با آن است.
\item
محیط\LTRfootnote{Enviroment}:درمرحله اجرا یا در مرحله کانفیگ
\item
پاسخ\LTRfootnote{Response} : 
سیستم باید ویژگی‌های مورد نیاز کاربر را فراهم کند یا با توجه به نیازهای کاربر آن را پیش‌بینی کند.
\item
اندازه گیری پاسخ\LTRfootnote{Response Measure}: یک یا چند مورد از این موارد را می‌توان به عنوان اندازه گیری استفاده نمود. زمان تسک، تعداد خطاها، تعداد تسک‌های تکمیل شده، میزان رضایت کاربر، درصد موفقیت‌آمیز عملیات‌ها نسبت به کل عملیات‌ها یا میان زمان یا داده از دست رفته زمانی که خطا رخ داده است.
\end{itemize}

محققان حوزه ارتباط انسان و کامپیوتر از اصطلاحات ابتکار کاربر \footnote{User Initiative} ، ابتکار سیستم و یا ابتکار درهم استفاده می‌کنند تا توصیف کنند که کدام جفت انسان-کامپیوتر ابتکار عمل خاصی را انجام می‌دهند و تعامل گونه پیش‌ می‌رود.

از تمایز بین ابتکار کاربر و سیستم برای بحث در مورد تاکتیک‌های قابلیت استفاده می‌توان بهره برد و سناریوهای مختلف را بررسی کرد.
هدف نهایی این تاکتیک‌ها ان است که به کاربر فیدبک و یاری مناسب داده شود.

دسته اول تاکتیک‌ها پشتیبانی ابتکار کاربر است. که می‌توان آن‌ها را به صورت زیر طبقه بندی کرد.
\begin{itemize}
\item
کنسل: سیستم باید به درخواست کنسل کردن گوش دهد و دستوری که کسنل شده متوقف شود و منابع آن ازاد گردد و مولفه‌های وابسته نیز اطلاع داده شوند.
\item
توقف/ادامه: سیستم در صورت توقف باید به صورت موقت منابع را آزاد کند که ممکن است بوسیله تسک‌های دیگر دوباره تخصیص داده شوند.
\item
بازگشت: باید این امکان وجود داشته باشد که یک یک حالت قبلی به درخواست کاربر بازگردانی شود.
\item
ترکیب: امکان ترکیب اشیا با سطح پایین به گروه. در این صورت کاربر می‌تواند یک عملیات را بر روی یک گروه اعمال کند و نه بر روی تک تک اشیا سطح پایین.
\end{itemize}

دسته دوم تاکتیک‌ها پشتیبانی ابتکار سیستم است. که می‌توان آن‌ها را به صورت زیر طبقه بندی کرد.
\begin{itemize}
\item
حفظ مدل تسک: زمینه را تعیین می‌کند تا سیستم بتواند از آنچه کاربر در تلاش است ایده ای داشته باشد و به او کمک کند.
\item
حفظ مدل کاربر: صریحا دانش کاربر از سیستم، رفتار کاربر از نظر زمان پاسخ مورد نظر و غیره را نشان می‌ٔهد.
\item
حفظ مدل سیستم: سیستم یک مدل صریح از خود را حفظ می‌کند. که از آن به منظور تعیین رفتار سیستم استفاده می‌شود تا برخورد مناسب به کاربر ارائه شود.
\end{itemize}

\section{طراحی فهرست بازبینی برای قابلیت استفاده}
\subsection{تخصیص مسئولیت ها} 

اطمینان حاصل شود که مسئولیت‌های اضافی در سیستم در صورت لزوم برای کمک به کاربر در موارد زیر در نظر گرفته شده است:
\begin{itemize}
\item
یادگیری نحوه استفاده از سیستم 
\item
دستیابی موثر به وطیفه موجود 
\item
تطبیق‌پذیری و پیکربندی سیستم
\item
بازیابی از خطاهای کاربر و سیستم
\end{itemize}

\subsection{مدل هماهنگی}

در این قسمت باید تعیین شود که عناصر سیستم بر چگونگی یادگیری کاربر برای استفاده از سیستم، کامل کردن تسک‌ها، تنظیم سیستم و بازیابی خطاها و سیستم، رضایت و اعتماد به نفس کاربر تاثیر می‌گذارد.
برای مثال  آیا می‌توان سیستم به رویدادهای حرکت کاربر بر روی منوها به خوبی پاسخ دهد و در زمان واقعی بازخورد مناسب بدهد؟
\subsection{مدل داده}
در این قسمت انتزاعات عمده داده را که با رفتار قابل درک کاربر درگیر هستند تعیین تعیین می‌شود.
باید اطمینان حاصل کرد که این انتزاعات اصلی داده ها ، عملکردهای آنها و خصوصیات آنها برای کمک به کاربر در دستیابی به وظیفه پیش رو، سازگاری و پیکربندی سیستم، بازیابی از خطاهای کاربر و سیستم، یادگیری نحوه استفاده از سیستم و افزایش رضایت کاربر طراحی شده است.
به عنوان مثال، انتزاعات داده باید به گونه ای طراحی شوند که از عملیات بازگشت و لغو پشتیبانی را پشتیبانی کند: و ریزدانگی این عملیات‌ها به گونه‌ای باشد که بیش از حد طولانی نشود.
\subsection{نقشه برداری در میان عناصر معماری}
در این مرحله باید تعیین شود که چه نقشه برداری از بین عناصر معماری برای کاربر نهایی قابل مشاهده است (به عنوان مثال میزان آگاهی کاربر نهایی از اینکه کدام سرویس ها محلی و کدام یک از راه دور هستند).
برای آن‌هایی که قابل مشاهده هستند، تعیین شود که این قابل مشاهده بودن چگونه بر روی سهولت یادگیری کاربر برای استفاده از سیستم، دستیابی به وظیفه پیش رو، سازگاری و پیکربندی سیستم، بازیابی از خطاهای کاربر و سیستم و اعتماد به نفس و رضایت کاربر تاثیر می‌گذارد.
\subsection{مدیریت منابع}
در این قسمت تعیین می‌شود که کاربر چگونه می تواند از منابع سیستم استفاده کند و پیکربندی را انجام دهد.
باید اطمینان حاصل شود که محدودیت های منابع تحت تنظیمات تمام کنترل شده توسط کاربر، احتمال رسیدن کاربران به وظایفشان را کمتر نخواهد کرد. به عنوان مثال، از تنظیماتی که منجر به زمان پاسخ بیش از حد طولانی می شود، جلوگیری باید کرد.
همچنین اطمینان حاصل شود که سطح منابع بر توانایی کاربران در یادگیری نحوه استفاده از سیستم تأثیر نگذارد، یا میزان اطمینان و رضایت آنها از سیستم را کاهش ندهد.
\subsection{زمان اتصال}
در این بخش تصمیمات مربوط به زمان  اتصال باید تحت کنترل کاربر باشد و اطمینان حاصل شود که کاربران می توانند تصمیماتی بگیرند که به قابلیت استفاده کمک می کند.
به عنوان مثال، اگر کاربر می تواند در زمان اجرا، پیکربندی سیستم یا پروتکل های ارتباطی آن یا عملکرد آن را از طریق افزونه ها انتخاب کند، باید اطمینان حاصل کرد که چنین گزینه هایی بر توانایی کاربر در یادگیری ویژگی های سیستم، استفاده موثر از سیستم، به حداقل رساندن تأثیر خطاها، سازگاری و پیکربندی بیشتر سیستم یا افزایش اطمینان و رضایت 
تأثیر منفی نمی گذارد.
\subsection{انتخاب فناوری}
انتخاب فناوری باید به گونه‌ای باشد که سناریو‌های قابلیت استفاده را بر روی سیستم بتوان اجرا کرد.
باید اطمینان پیدا کرد که تکنولوژی انتخاب شده تاثیر برعکس ندارد و به ویژگی‌های یادگیری سیستم خللی وارد نمی‌کند. 


\section{مطالعات موردی}

در اینجا مطالعات موردی در ۳ حوزه حمل و نقل درون شهری برای نیاز قابلیت استفاده انجام شده است. 
\subsection{تاکسیرانی و حمل‌و‌نقل درون شهری}

اپلیکیشن MyWay :
اپلیکیشن OLA : 
\subsection{اشتراک خودرو و حمل‌و‌نقل درون شهری}
