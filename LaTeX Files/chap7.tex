% Chapter 7
\chapter{قابلیت استفاده}
\section{تعریف قابلیت استفاده}
قابلیت استفاده\LTRfootnote{Usability}
به این تعریف می‌پردازد که چقدر برای کار انجام دادن یک وظیفه دلخواه راحت است. این ویژگی یکی از ارزان و راحت‌ترین ویژگی‌هایی است که می‌توان به وسیله آن کیفیت سیستم را افزایش داد. 
قابلیت استفاده شامل موارد زیر است:
\begin{itemize}
\item
ویژگی‌ها سیستم یادگیری
\item
استفاده موثر از یک سیستم 
\item
به حداقل رساندن تاثیر خطاها
\item
تطبیق سیستم با نیازهای کاربر
\item
افزایش اعتماد به نفس و رضایت
\end{itemize}


\section{سناریوی عمومی قابلیت استفاده}
سناریو‌ی عمومی قابلیت استفاده به صورت زیر است:
\begin{itemize}
\item
منبع محرک\LTRfootnote{Source of stimulus} : کاربر نهایی
\item
محرک\LTRfootnote{Stimulus} : کاربر نهایی که سعی می‌کند از سیستم به طور موثر استفاده کند و آن را یاد بگیرد و تاثیر خطاها را کمینه کند، با سیستم تطبیق پذیرد و آن را تنظیم کند.
\item
محصول\LTRfootnote{Artifact}: سیستم یا بخشی از سیستم که کاربر در حال تعامل با آن است.
\item
محیط\LTRfootnote{Enviroment}:درمرحله اجرا یا در مرحله کانفیگ
\item
پاسخ\LTRfootnote{Response} : 
سیستم باید ویژگی‌های مورد نیاز کاربر را فراهم کند یا با توجه به نیازهای کاربر آن را پیش‌بینی کند.
\item
اندازه گیری پاسخ\LTRfootnote{Response Measure}: یک یا چند مورد از این موارد را می‌توان به عنوان اندازه گیری استفاده نمود. زمان تسک، تعداد خطاها، تعداد تسک‌های تکمیل شده، میزان رضایت کاربر، درصد موفقیت‌آمیز عملیات‌ها نسبت به کل عملیات‌ها یا میان زمان یا داده از دست رفته زمانی که خطا رخ داده است.
\end{itemize}

محققان حوزه ارتباط انسان و کامپیوتر از اصطلاحات ابتکار کاربر \footnote{User Initiative} ، ابتکار سیستم و یا ابتکار درهم استفاده می‌کنند تا توصیف کنند که کدام جفت انسان-کامپیوتر ابتکار عمل خاصی را انجام می‌دهند و تعامل گونه پیش‌ می‌رود.

از تمایز بین ابتکار کاربر و سیستم برای بحث در مورد تاکتیک‌های قابلیت استفاده می‌توان بهره برد و سناریوهای مختلف را بررسی کرد.
هدف نهایی این تاکتیک‌ها ان است که به کاربر فیدبک و یاری مناسب داده شود.

دسته اول تاکتیک‌ها پشتیبانی ابتکار کاربر است. که می‌توان آن‌ها را به صورت زیر طبقه بندی کرد.
\begin{itemize}
\item
کنسل: سیستم باید به درخواست کنسل کردن گوش دهد و دستوری که کسنل شده متوقف شود و منابع آن ازاد گردد و مولفه‌های وابسته نیز اطلاع داده شوند.
\item
توقف/ادامه: سیستم در صورت توقف باید به صورت موقت منابع را آزاد کند که ممکن است بوسیله تسک‌های دیگر دوباره تخصیص داده شوند.
\item
بازگشت: باید این امکان وجود داشته باشد که یک یک حالت قبلی به درخواست کاربر بازگردانی شود.
\item
ترکیب: امکان ترکیب اشیا با سطح پایین به گروه. در این صورت کاربر می‌تواند یک عملیات را بر روی یک گروه اعمال کند و نه بر روی تک تک اشیا سطح پایین.
\end{itemize}

دسته دوم تاکتیک‌ها پشتیبانی ابتکار سیستم است. که می‌توان آن‌ها را به صورت زیر طبقه بندی کرد.
\begin{itemize}
\item
حفظ مدل تسک: زمینه را تعیین می‌کند تا سیستم بتواند از آنچه کاربر در تلاش است ایده ای داشته باشد و به او کمک کند.
\item
حفظ مدل کاربر: صریحا دانش کاربر از سیستم، رفتار کاربر از نظر زمان پاسخ مورد نظر و غیره را نشان می‌ٔهد.
\item
حفظ مدل سیستم: سیستم یک مدل صریح از خود را حفظ می‌کند. که از آن به منظور تعیین رفتار سیستم استفاده می‌شود تا برخورد مناسب به کاربر ارائه شود.
\end{itemize}

\section{طراحی فهرست بازبینی برای قابلیت استفاده}
\subsection{تخصیص مسئولیت ها} 
\subsection{مدل هماهنگی}
\subsection{مدل داده}
\subsection{نقشه برداری در میان عناصر معماری}
\subsection{مدیریت منابع}
\subsection{زمان اتصال}
\subsection{انتخاب فناوری}
