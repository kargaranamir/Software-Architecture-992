% Chapter 3
\chapter{اصلاح‌پذیری}
\section{تعریف اصلاح‌پذیری}
اصلاح‌پذیری
\footnote{Modifiability}
یا همان قابلیت اصلاح یکی از موارد مهمی است که تقریبا در تمامی مدل‌های کیفیت نرم‌افزار مورد سنجش قرار می‌گیرد. در این نیازمندی غیرعملکردی به اهمیت اصلاح‌پذیری نرم‌افزار با هزینه کم پرداخته می‌شود.

برای اصلاح‌پذیری تعاریف زیادی می‌توان ارائه نمود که از جمله‌آن‌ها می‌توان به موارد زیر اشاره کرد \cite{mod:c1-cuibancan-2020}.

\begin{enumerate}
\item
چقدر نرم‌افزار قابل تغییر است؟
\item 
هزینه تغییر 
\item
میزان سهولت انجام تغییر
\end{enumerate}

مطالعات زیادی نشان داده است که با پرداختن به ویژگی اصلاح‌پذیری می‌توان در حین توسعه به طور چشم‌گیر هزینه یک سیستم‌ نرم‌افزاری را کاهش داد.
همچنین به موجب پرداختن به این ویژگی به معمار نرم‌افزار این امکان داده می‌شود که هزینه‌های نگهداری را پیش‌بینی و ریسک و انعطاف پذیری نرم‌افزار را ارزیابی کند.

\section{سناریوی عمومی اصلاح‌پذیری}

سناریو‌ی عمومی قابلیت اصلاح‌پذیری به صورت زیر است:
\begin{itemize}
\item
منبع محرک\LTRfootnote{Source of stimulus} : هر یک از کاربران، توسعه دهندگان، ادمین‌ها
\item
محرک\LTRfootnote{Stimulus} : اضافه، حذف یا مورد تغییر قرار دادن یک کارایی، یا تغییر دادن یک ویژگی کیفیت یا ظرفیت یا تکنولوژی.
\item
محصول\LTRfootnote{Artifact}:  کد، داده، مولفه‌ها و ...
\item
محیط\LTRfootnote{Enviroment}: در زمان کامپایل، زمان اجرا، زمان ساخت، زمان طراحی
\item
پاسخ\LTRfootnote{Response} : در پاسخ مناسب یک یا چند عملیات رخ می‌دهد. 
در این عملیات‌ها انجام و تست و استقرار تغییر انجام می‌شود.
\item
اندازه گیری پاسخ\LTRfootnote{Response Measure}: برای اندازه‌گیری پاسخ از معیار‌‌های هزینه و تعداد و سایز و پیچیدگی موارد مورد بررسی، تلاش و .. استفاده می‌شود.
\end{itemize}

 تاکتیک‌های زیادی برای بهبود این معیار معرفی شده است که هدف همه آن‌ها کنترل پیچیدگی تغییرات و انجام تغییرات، تست و استقرار آن‌ها در زمان و هزینه تعیین شده است.
به صورت عمومی می‌توان از بیشتر ‌آن‌ها در سامانه‌های نرم‌افزاری حمل و نقل تاکسی آنلاین استفاده نمود.

یکی از این تاکتیک‌ها کم کردن حجم ماژول‌های برنامه است. اگر بتوان یک ماژول که را به ماژول‌های کوچکتر تبدیل کرد، آنگاه در این صورت هزینه تغییرات در آینده نیز به صورت میانگین کاهش پیدا می‌کند.
چرا که هرچه حجم ماژول بیشتر باشد هزینه تغییر آن زیادتر است.

تاکتیک دیگر افزایش مقدار چسبندگی
\footnote{\lr{cohesion}}
است. به این صورت که اگر دو وظیفه داخل یک ماژول یک هدف را دنبال نمی‌کنند آنگاه باید آن دو در ماژول‌های متفاوتی قرار بگیرند.
این کار ممکن است موجب ایجاد ماژول‌های جدید یا جابه‌جایی وظایف به ماژول‌های موجود شود.

همچنین بر اساس اصل \lr{loose coupling} تلاش می‌شود که تا حد ممکن وابستگی‌ها ماژول‌ها کاهش یاید و آن‌ها را محدود کرد. عملیات refactor را نیز بر اساس همین اصل می‌توان زمانی که دو ماژول تحت تاثیر یک تغییر، تاثیر می‌پذیرند انجام داد. چرا که در واقع قسمتی از آن دو کپی یکدیگر هستند.

و در آخر باید گفت که هر چه عملیات اتصال را بتوان به تعویق بیندازیم تا انعطاف‌پذیری تغییر هر جز حفظ شود و هزینه تغییر خاص هر کدام کاهش یابد بهتر است.


به منظور بهره بری از این تاکتیک‌ها می‌توان با توجه به پروژه نرم‌افزاری خاص یک چک لیست داشت که در اینجا با توجه به خصوصیت‌های سامانه‌های نرم‌افزاری حمل و نقل تاکسی آنلاین به آن‌ها پاسخ داده می‌شود.

\section{طراحی فهرست بازبینی برای اصلا‌ح‌پذیری}
\subsection{تخصیص مسئولیت ها} 
تعیین شود که با توجه به تغییر در هر یک از بخش‌های فنی، قانونی، اجتماعی، تجاری چه تغییراتی احتمال دارد رخ دهد. برای هر تغیر بالقوه مسئولیت‌هایی تعیین شود که برای ایجاد تغییر نیاز به افزودن، اصلاح یا حذف دارند.
همچنین مسئولیت‌هایی که تغییر می‌کنند باید در یک ماژول قرار بگیرند و مسئولیت‌هایی که در زمان‌های مختلف تغییر می کنند در ماژولهای جداگانه قرار بگیرند.

\subsection{مدل هماهنگی}

در این قسمت تعیین شود که کدام ویژگی یا ویژگی کیفیت می تواند در زمان اجرا تغییر کند و این موضوع چگونه بر هماهنگی تأثیر می گذارد. 
به عنوان مثال، آیا اطلاعات منتقل شده در زمان اجرا تغییر می کند؟ یا پروتکل ارتباطی در زمان اجرا تغییر می کند؟
در این صورت، اطمینان حاصل شود که چنین تغییراتی تعداد کمی از ماژول‌ها را تحت تأثیر قرار دهد.

مشخص شود که کدام دستگاه‌ها، پروتکل‌ها و مسیرهای ارتباطی مورد استفاده برای هماهنگی محتما است که تغییر کنند. برای آن دستگاه ها، پروتکل ها و مسیرهای ارتباطی ، اطمینان حاصل شود که تأثیر تغییرات فقط به مجموعه کوچکی از ماژول ها محدود می شود.

برای آن دسته از عناصر که اصلاح‌پذیری دغدغه است، از یک مدل هماهنگی استفاده شود که اتصال را کاهش دهد، 
اتصالات مانند گذرگاه خدمات سازمانی
\footnote{\lr{enterpise service bus}} 
به تعویق افتد یا وابستگی‌هایی مانند بخش\footnote{broadcast}
محدود شود.
\subsection{مدل داده}
در این قسمت تعیین می‌شود که چه تغییراتی (یا دسته بندی تغییرات) در انتزاع داده ها، عملکردهای آنها یا خصوصیات آنها محتمل است که رخ دهد.

همچنین تعیین می‌‌شود که کدام تغییر یا دسته تغییرات در این انتزاعات داده شامل ایجاد، شروع اولیه، ماندگاری، دستکاری، ترجمه یا تخریب آنها است.

برای هر تغییر یا دسته تغییر، تعیین شود که آیا تغییرات توسط کاربر نهایی، مدیر سیستم یا توسعه دهنده ایجاد می شود. برای تغییرات ایجاد شده توسط کاربر نهایی یا ادمین، اطمینان حاصل شود که ویژگی های لازم برای آن کاربر قابل مشاهده است و کاربر دارای مجوزهای صحیح برای اصلاح داده‌ها، عملکردهای آن یا خصوصیات آن است.

برای هر تغییر بالقوه یا دسته تغییر:

\begin{itemize}
\item
تعیین شود که کدام انتزاعات داده باید اضافه، اصلاح یا حذف شوند.
\item
تعیین شود که آیا تغییراتی در ایجاد، مقداردهی اولیه، ماندگاری، دستکاری، ترجمه یا تخریب این انتزاعات داده ایجاد می شود یا خیر.
\item
تعیین شود که سایر انتزاعات داده تحت تأثیر تغییر قرار می گیرند.
برای این انتزاعات اضافی، تعیین شود که آیا تأثیر بر عملکرد، خصوصیات، ایجاد، شروع اولیه، ماندگاری، دستکاری، ترجمه یا تخریب آن‌ها خواهد بود.
\item
اطمینان حاصل شود که 
تخصیص انتزاع داده‌ها تعداد تغییرات بالقوه و شدت اصلاحات در انتزاعات را به حداقل می رساند.
\end{itemize}
مدل داده به گونه‌ای طراحی شود که موارد اختصاص یافته به هر عنصر از مدل داده احتمالاً با یکدیگر تغییر پیدا کنند.

\subsection{نقشه برداری در میان عناصر معماری}
در این قسمت تعیین می‌شود که آیا تغییر نحوه نگاشت قابلیت به عناصر محاسباتی (به عنوان مثال فرآیندها ، رشته‌ها ، پردازنده ها) در زمان اجرا، زمان کامپایل، زمان طراحی یا زمان ساخت مطلوب است.
میزان تغییرات لازم برای افزودن، حذف یا اصلاح یک تابع یا یک ویژگی کیفیت را تعیین شود. که این تعیین این ممکن است شامل تعیین موارد زیر باشد، به عنوان مثال:
\begin{itemize}
\item
اجرای وابستگی‌ها
\item
اختصاص داده به پایگاه داده
\item
اختصاص عناصر زمان اجرا به پردازش ها‌، رشته‌ها یا پردازنده‌ها
\end{itemize}
اطمینان حاصل شود که چنین تغییراتی با مکانیزمی که اتصال را به تاخیر می‌اندازد انجام شود.

\subsection{مدیریت منابع}
در این قسمت تعیین می‌شود که چگونه اضافه کردن، حذف یا تغییر یک مسئولیت یا ویژگی کیفیت بر استفاده از منابع تأثیر می گذارد.
به عنوان مثال: 
\begin{itemize}
\item
تعیین اینکه چه تغییراتی ممکن است منابع جدید را معرفی کند یا منابع قدیمی را حذف کند یا بر استفاده از منابع موجود تأثیر بگذارد.
\item
تعیین اینکه چه محدودیت‌های منابعی تغییر می کنند و چگونه باید اطمینان حاصل کرد که منابع پس از اصلاح برای تأمین نیازهای سیستم کافی هستند.
\item
همه مدیریت‌کنندگان منابع کپسوله شوند و اطمینان حاصل شود که سیاست های اعمال شده توسط مدیران منابع ، خود کپسوله شده و اتصال‌ها تا حد ممکن به تعویق می افتند.
\end{itemize}

\subsection{زمان اتصال}
برای هر تغییر یا دسته تغییرات:
\begin{itemize}
\item
آخرین زمانی را که باید تغییر ایجاد شود تعیین شود.
\item
مکانیزم تعویق اتصال که توانایی مناسب را در زمان انتخاب شده ارائه می‌دهد انتخاب شود.
\item
هزینه معرفی مکانیسم و هزینه ایجاد تغییرات را با استفاده از مکانیزم انتخاب شده تعیین شود.
\item
تعداد زیادی انتخاب اتصال معرفی نشود که مانع تغییر شود زیرا وابستگی ها در بین گزینه ها پیچیده و ناشناخته است.
\end{itemize}
\subsection{انتخاب فناوری}
با توجه به انتخاب‌های فناوری تعیین کنید که چه اصلاحاتی سخت‌تر و چه اصلاحاتی آسان‌تر خواهد بود. 
آیا انتخاب‌های به انجام و تست و استقرار تغییرات کمک‌ می‌کند؟
چقدر تغییر انتخاب خود فناوری راحت است؟ فناوری‌ها را باید به گونه‌ای انتخاب کرد بیشترین پشتیبانی از تغییرات را داشته باشد.




