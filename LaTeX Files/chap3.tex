\chapter{اصلاح‌پذیری}
\section{تعریف اصلاح‌پذیری}
اصلاح‌پذیری
\footnote{Modifiability}
یا همان قابلیت اصلاح یکی از موارد مهمی است که تقریبا در تمامی مدل‌های کیفیت نرم‌افزار مورد سنجش قرار می‌گیرد. در این نیازمندی غیرعملکردی به اهمیت اصلاح‌پذیری نرم‌افزار با هزینه کم پرداخته می‌شود.

برای اصلاح‌پذیری تعاریف زیادی می‌توان ارائه نمود که از جمله‌آن‌ها می‌توان به موارد زیر اشاره کرد \cite{mod:c1-cuibancan-2020}.

\begin{enumerate}
\item
چقدر نرم‌افزار قابل تغییر است؟
\item 
هزینه تغییر 
\item
میزان سهولت انجام تغییر
\end{enumerate}

مطالعات زیادی نشان داده است که با پرداختن به ویژگی اصلاح‌پذیری می‌توان در حین توسعه به طور چشم‌گیر هزینه یک سیستم‌ نرم‌افزاری را کاهش داد.
همچنین به موجب پرداختن به این ویژگی به معمار نرم‌افزار این امکان داده می‌شود که هزینه‌های نگهداری را پیش‌بینی و ریسک و انعطاف پذیری نرم‌افزار را ارزیابی کند.

در یک سناریو عمومی یک پروژه نرم‌افزاری هر یک از کاربران، توسعه دهندگان، ادمین‌ها می‌توانند به عنوان یک منبع 
\footnote{source}
موجب ایجاد
یک انگیختار 
\footnote{stimulus}
شوند. آن انگیختار می‌تواند یک کارایی را اضافه حذف یا مورد تغییر قرار داده، یا یک ویژگی کیفیت یا ظرفیت یا تکنولوژی را تغییر دهد.
که مورد بررسی آن می‌تواند بر کد، داده، مولفه‌ها و ... باشد و محیط آن می‌تواند در زمان کامپایل، زمان اجرا، زمان ساخت، زمان طراحی و ... باشد.

در پاسخ مناسب یک یا چند عملیات رخ می‌دهد. 
در این عملیات‌ها انجام و تست و استقرار تغییر انجام می‌شود.
برای اندازه‌گیری پاسخ از معیار‌‌های هزینه و تعداد و سایز و پیچیدگی موارد مورد بررسی، تلاش و .. استفاده می‌شود.
 
 تاکتیک‌های زیادی برای بهبود این معیار معرفی شده است که هدف همه آن‌ها کنترل پیچیدگی تغییرات و انجام تغییرات، تست و استقرار آن‌ها در زمان و هزینه تعیین شده است.
 
 \section{اصلاح‌پذیری در عمل}
 
 
 