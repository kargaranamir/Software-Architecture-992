\chapter{اصلاح‌پذیری}
\section{تعریف اصلاح‌پذیری}
اصلاح‌پذیری
\footnote{Modifiability}
یا همان قابلیت اصلاح یکی از موارد مهمی است که تقریبا در تمامی مدل‌های کیفیت نرم‌افزار مورد سنجش قرار می‌گیرد. در این نیازمندی غیرعملکردی به اهمیت اصلاح‌پذیری نرم‌افزار با هزینه کم پرداخته می‌شود.

برای اصلاح‌پذیری تعاریف زیادی می‌توان ارائه نمود که از جمله‌آن‌ها می‌توان به موارد زیر اشاره کرد \cite{mod:c1-cuibancan-2020}.

\begin{enumerate}
\item
چقدر نرم‌افزار قابل تغییر است؟
\item 
هزینه تغییر 
\item
میزان سهولت انجام تغییر
\end{enumerate}

مطالعات زیادی نشان داده است که با پرداختن به ویژگی اصلاح‌پذیری می‌توان در حین توسعه به طور چشم‌گیر هزینه یک سیستم‌ نرم‌افزاری را کاهش داد.
همچنین به موجب پرداختن به این ویژگی به معمار نرم‌افزار این امکان داده می‌شود که هزینه‌های نگهداری را پیش‌بینی و ریسک و انعطاف پذیری نرم‌افزار را ارزیابی کند.

در یک سناریو عمومی یک پروژه نرم‌افزاری هر یک از کاربران، توسعه دهندگان، ادمین‌ها می‌توانند به عنوان یک منبع 
\footnote{source}
موجب ایجاد
یک انگیختار 
\footnote{stimulus}
شوند. آن انگیختار می‌تواند یک کارایی را اضافه حذف یا مورد تغییر قرار داده، یا یک ویژگی کیفیت یا ظرفیت یا تکنولوژی را تغییر دهد.
که مورد بررسی آن می‌تواند بر کد، داده، مولفه‌ها و ... باشد و محیط آن می‌تواند در زمان کامپایل، زمان اجرا، زمان ساخت، زمان طراحی و ... باشد.

در پاسخ مناسب یک یا چند عملیات رخ می‌دهد. 
در این عملیات‌ها انجام و تست و استقرار تغییر انجام می‌شود.
برای اندازه‌گیری پاسخ از معیار‌‌های هزینه و تعداد و سایز و پیچیدگی موارد مورد بررسی، تلاش و .. استفاده می‌شود.
 
 تاکتیک‌های زیادی برای بهبود این معیار معرفی شده است که هدف همه آن‌ها کنترل پیچیدگی تغییرات و انجام تغییرات، تست و استقرار آن‌ها در زمان و هزینه تعیین شده است.
 
 \section{اصلاح‌پذیری در عمل}
در عمل‌ تاکتیک‌های متنوعی برای بهبود این معیار وجود دارد. که به صورت عمومی می‌توان از بیشتر ‌آن‌ها در سامانه‌های نرم‌افزاری حمل و نقل تاکسی آنلاین استفاده نمود.

یکی از این تاکتیک‌ها کم کردن حجم ماژول‌های برنامه است. اگر بتوان یک ماژول که را به ماژول‌های کوچکتر تبدیل کرد، آنگاه در این صورت هزینه تغییرات در آینده نیز به صورت میانگین کاهش پیدا می‌کند.
چرا که هرچه حجم ماژول بیشتر باشد هزینه تغییر آن زیادتر است.

تاکتیک دیگر افزایش مقدار چسبندگی
\footnote{\lr{cohesion}}
است. به این صورت که اگر دو وظیفه داخل یک ماژول یک هدف را دنبال نمی‌کنند آنگاه باید آن دو در ماژول‌های متفاوتی قرار بگیرند.
این کار ممکن است موجب ایجاد ماژول‌های جدید یا جابه‌جایی وظایف به ماژول‌های موجود شود.

همچنین بر اساس اصل \lr{loose coupling} تلاش می‌شود که تا حد ممکن وابستگی‌ها ماژول‌ها کاهش یاید و آن‌ها را محدود کرد. عملیات refactor را نیز بر اساس همین اصل می‌توان زمانی که دو ماژول تحت تاثیر یک تغییر، تاثیر می‌پذیرند انجام داد. چرا که در واقع قسمتی از آن دو کپی یکدیگر هستند.

و در آخر باید گفت که هر چه عملیات اتصال را بتوان به تعویق بیندازیم تا انعطاف‌پذیری تغییر هر جز حفظ شود و هزینه تغییر خاص هر کدام کاهش یابد بهتر است.


به منظور بهره بری از این تاکتیک‌ها می‌توان با توجه به پروژه نرم‌افزاری خاص یک چک لیست داشت که در اینجا با توجه به خصوصیت‌های سامانه‌های نرم‌افزاری حمل و نقل تاکسی آنلاین به آن‌ها پاسخ داده می‌شود.

\section{چک لیست اصلاح پذیری}

\subsection{تخصیص مسئولیت‌ها}
تعیین شود که با توجه به تغییر در هر یک از بخش‌های فنی، قانونی، اجتماعی، تجاری چه تغییراتی احتمال دارد رخ دهد. برای هر تغیر بالقوه مسئولیت‌هایی را تعیین شود 
که برای ایجاد تغییر نیاز به افزودن، اصلاح یا حذف دارند.
همچنین مسئولیت‌هایی را که تغییر می‌کنند باید در یک ماژول قرار بگیرند و مسئولیت‌هایی که در زمان‌های مختلف تغییر می کنند در ماژولهای جداگانه قرار بگیرند.
