% Chapter 7 
\chapter{قابلیت آزمون}
\section{تعریف قابلیت آزمون}
طبق تخمین ها بین ۳۰ تا ۵۰ درصد هزینه تولید یک سیستم به آزمون سیستم اختصاص دارد.قابلیت آزمون\LTRfootnote{Testability} به سهولت ساخت نرم افزار برای نشان دادن عیب های خود از طریق آزمون اشاره دارد. به طور خاص ، قابلیت آزمون احتمال آن است که اگر برنامه دارای خطایی باشد،‌در آزمون خطا خود را نشان دهد. به صورت شهودی ، اگر یک سیستم به راحتی خطا های خود را آشکار سازد ،‌دارای قابلیت آزمون است

معیارهای پاسخ دهی برای قابلیت آزمون به میزان موثر بودن آزمون ها در کشف عیب ها و مدت زمان انجام آزمایش ها تا رسیدن به حد مطلوبی از پوشش\LTRfootnote{Coverage} مربوط می شوند.

\section{سناریوی عمومی قابلیت آزمون}
سناریو‌ی عمومی قابلیت استفاده به صورت زیر است:
\begin{itemize}
\item
منبع محرک\LTRfootnote{Source of stimulus} : هر کسی که برای سیستم آزمون طراحی می کند می تواند جز محرک های سناریو عمومی باشد.آزمون های طراحی شده می توانند به صورت دستی یا اتوماتیک اجرا شوند.
\item
محرک\LTRfootnote{Stimulus} : اجرای مجموعه ی خاصی از آزمون ها که می تواند به یکی از دلایل اضافه شدن مولفه ای جدید یا تحویل به مشریان صورت پذیرفته باشد.
\item
محصول\LTRfootnote{Artifact}: سیستم یا بخشی از آن که مورد آزمون قرار می گیرد.
\item
محیط\LTRfootnote{Enviroment}:: زمان طراحی ، زمان توسعه ، زمان کامپایل ، زمان ادغام\LTRfootnote{Integration Time} ، زمان استقرار\LTRfootnote{Deployment Time} ، زمان اجرا
\item
پاسخ\LTRfootnote{Response} : 
پاسخ سیستم می تواند زیر مجموعه ای از موارد زیر باشد.
\begin{itemize}
\item
اجرای مجموعه ی آزمون ها و دریافت پاسخ آن ها
\item
جمع آوری فعالیت هایی که به خطا در آزمون منجر شده اند
\item
کنترل و نظارت بر وضعیت سیستم
\end{itemize}

\item
اندازه گیری پاسخ\LTRfootnote{Response Measure}: یک یا چند مورد از این موارد را می‌توان به عنوان اندازه گیری استفاده نمود.
\begin{itemize}
\item
تلاش برای یافتن یک عیب یا کلاسی از عیب ها
\item
تلاش برای دستیابی به درصد معینی از پوشش فضا\LTRfootnote{Space Coverage}
\item
احتمال بروز خطا در آزمون بعدی
\item
زمان انجام آزمون ها
\item
طول طولانی ترین زنجیره وابستگی در آزمون
\item
مدت زمان آماده سازی محیط آزمون
\end{itemize}
\end{itemize}

\section{تاکتیک ها در قابلیت آزمون}
هدف تاکتیک های قابلیت آزمون آسان تر ساختن انجام آزمون ها پس از پایان فرآیند توسعه اولیه سیستم است.روش هایی که برای افزایش قابلیت آزمون وجود دارند در دو دسته از تاکتیک های قابلیت آزمون قرار می گیرند؛‌دسته ی اول به قابلیت کنترل و نظارت بر سیستم می افزایند و از این طریق انجام آزمون ها را آسان تر می سازند و دسته ی دیگر بر محدود کردن پیچیدگی های سیستم تمرکز می کنند.
\subsection{کنترل و نظارت بر وضعیت سیستم}
ساده ترین شکل کنترل و نظارت بر یک سیستم نرم افزاری با مجموعه ای از ورودی ها است.اجازه می دهید سیستم کار خود با ورودی های داده شده را انجام دهد و سپس خروجی آن را مشاهده می کنید.به صورت خاص روش هایی که در این دسته بندی وجود دارند شامل موارد زیر هستند:
\begin{itemize}
\item
رابط های تخصصی\LTRfootnote{Specialized interfaces} : 
\item
ضبط/پخش \LTRfootnote{Record/Playback} :
\item
محلی سازی ذخیره سازی \LTRfootnote{Localize state storage} :
\item
منابع داده انتزاعی \LTRfootnote{Abstract data sources} :
\item
جعبه شنی \LTRfootnote{Sandbox} :
\item
ادعاهای قابل اجرا \LTRfootnote{Executable assertions} :
\end{itemize}
\subsection{محدودسازی پیچیدگی های سیستم}
هر چقدر یک نرم افزار پیچیده تر باشد، آزمون آن سخت تر خواهد بود زیرا بنا بر تعریف پیچیدگی ، فضای حالت عملیاتی یک برنامه ی پیچیده بسیار بزرگ است و ایجاد مجدد یک حالت خاص در یک فضای حالت بزرگ دشوارتر از انجام این کار در فضای حالت کوچک است.به طور خاص دو روش زیر به محدودسازی پیچیدگی نرم افزار کمک می کنند.
\begin{itemize}
\item
محدود‌سازی پیچیدگی ساختاری :
\item
دوری از عدم قطعیت \LTRfootnote{Nondeterminism} :
\end{itemize}
\section{طراحی فهرست بازبینی برای قابلیت استفاده}
\subsection{تخصیص مسئولیت ها} 

اطمینان حاصل شود که مسئولیت‌های اضافی در سیستم در صورت لزوم برای‍ کمک به کاربر در موارد زیر در نظر گرفته شده است:
\begin{itemize}
\item
یادگیری نحوه استفاده از سیستم 
\item
دستیابی موثر به وطیفه موجود 
\item
تطبیق‌پذیری و پیکربندی سیستم
\item
بازیابی از خطاهای کاربر و سیستم
\end{itemize}

\subsection{مدل هماهنگی}

در این قسمت باید تعیین شود که عناصر سیستم بر چگونگی یادگیری کاربر برای استفاده از سیستم، کامل کردن تسک‌ها، تنظیم سیستم و بازیابی خطاها و سیستم، رضایت و اعتماد به نفس کاربر تاثیر می‌گذارد.
برای مثال آیا می‌توان سیستم به رویدادهای حرکت کاربر بر روی منوها به خوبی پاسخ دهد و در زمان واقعی بازخورد مناسب بدهد؟
\subsection{مدل داده}
در این قسمت انتزاعات عمده داده را که با رفتار قابل درک کاربر درگیر هستند تعیین تعیین می‌شود.
باید اطمینان حاصل کرد که این انتزاعات اصلی داده ها ، عملکردهای آنها و خصوصیات آنها برای کمک به کاربر در دستیابی به وظیفه پیش رو، سازگاری و پیکربندی سیستم، بازیابی از خطاهای کاربر و سیستم، یادگیری نحوه استفاده از سیستم و افزایش رضایت کاربر طراحی شده است.
به عنوان مثال، انتزاعات داده باید به گونه ای طراحی شوند که از عملیات بازگشت و لغو پشتیبانی را پشتیبانی کند: و ریزدانگی این عملیات‌ها به گونه‌ای باشد که بیش از حد طولانی نشود.
\subsection{نقشه برداری در میان عناصر معماری}
در این مرحله باید تعیین شود که چه نقشه برداری از بین عناصر معماری برای کاربر نهایی قابل مشاهده است (به عنوان مثال میزان آگاهی کاربر نهایی از اینکه کدام سرویس ها محلی و کدام یک از راه دور هستند).
برای آن‌هایی که قابل مشاهده هستند، تعیین شود که این قابل مشاهده بودن چگونه بر روی سهولت یادگیری کاربر برای استفاده از سیستم، دستیابی به وظیفه پیش رو، سازگاری و پیکربندی سیستم، بازیابی از خطاهای کاربر و سیستم و اعتماد به نفس و رضایت کاربر تاثیر می‌گذارد.
\subsection{مدیریت منابع}
در این قسمت تعیین می‌شود که کاربر چگونه می تواند از منابع سیستم استفاده کند و پیکربندی را انجام دهد.
باید اطمینان حاصل شود که محدودیت های منابع تحت تنظیمات تمام کنترل شده توسط کاربر، احتمال رسیدن کاربران به وظایفشان را کمتر نخواهد کرد. به عنوان مثال، از تنظیماتی که منجر به زمان پاسخ بیش از حد طولانی می شود، جلوگیری باید کرد.
همچنین اطمینان حاصل شود که سطح منابع بر توانایی کاربران در یادگیری نحوه استفاده از سیستم تأثیر نگذارد، یا میزان اطمینان و رضایت آنها از سیستم را کاهش ندهد.
\subsection{زمان اتصال}
در این بخش تصمیمات مربوط به زمان اتصال باید تحت کنترل کاربر باشد و اطمینان حاصل شود که کاربران می توانند تصمیماتی بگیرند که به قابلیت استفاده کمک می کند.
به عنوان مثال، اگر کاربر می تواند در زمان اجرا، پیکربندی سیستم یا پروتکل های ارتباطی آن یا عملکرد آن را از طریق افزونه ها انتخاب کند، باید اطمینان حاصل کرد که چنین گزینه هایی بر توانایی کاربر در یادگیری ویژگی های سیستم، استفاده موثر از سیستم، به حداقل رساندن تأثیر خطاها، سازگاری و پیکربندی بیشتر سیستم یا افزایش اطمینان و رضایت 
تأثیر منفی نمی گذارد.
\subsection{انتخاب فناوری}
انتخاب فناوری باید به گونه‌ای باشد که سناریو‌های قابلیت استفاده را بر روی سیستم بتوان اجرا کرد.
باید اطمینان پیدا کرد که تکنولوژی انتخاب شده تاثیر برعکس ندارد و به ویژگی‌های یادگیری سیستم خللی وارد نمی‌کند. 