% Chapter 2
\chapter{قابلیت همکاری}
قابلیت همکاری \LTRfootnote{Interoperability} نیازمندی ای است که به میزان توانایی دو یا بیشتر سیستم در انتقال معنادار اطلاعات از طریق رابط \LTRfootnote{Interface} هایشان در یک زمینه ی خاص می پردازد.این تعریف علاوه بر توانایی مبادله ی صحیح اطلاعات که به قابلیت همکاری نحوی \LTRfootnote{Syntatic Interoperability} منجر می شود به بعد فهم صحیح اطلاعات مبادله شده توسط طرفین یا به عبارتی قابلیت همکاری معنایی \LTRfootnote{semantic Interoperability} نیز تاکید ویژه ای دارد.

زمانی که از قابلیت همکاری صحبت به میان می آید وجود یک سیستم در انزوا معنا نمی دهد فلذا باید در مورد سیستم های همکار، نحوه ی همکاری و همچنین شرایطی که همکاری تحت آن صورت می گیرند تمامی موارد به خوبی مشخص شوند.

از جمله دلایلی که قابلیت همکاری در میان چندین سیستم را دارای اهمیت می سازند،‌دو مورد زیر مطرح هستند:
\begin{enumerate}
\item
سیستم طراحی شده توسط شما قرار است خدماتی را در اختیار سایر سیستم های از پیش شناخته شده یا نشده قرار دهد.برای مثال \lr{Google Maps} سیستمی است که خدمات زیادی را به سایر سیستم ها از طریق رابط های خود ارائه می دهد.
\item
سیستم شما متشکل از زیر سیستم هایی است که هر یک وظایف مشخصی را بر عهده دارند و همچنین این زیر سیستم ها به یک دیگر وابسته بوده و خدماتی را به یکدیگر ارائه می دهند.
\end{enumerate}

فرآیند همکاری دو مهم کشف\LTRfootnote{Discovery} و بکارگیری پاسخ ها \LTRfootnote{Handling of the response} را در بر دارد.مشتریان یک سیستم پیش از و یا در زمان همکاری با یک سیستم نیاز دارند تا از مکان\LTRfootnote{Location}‌،اطلاعات هویتی\LTRfootnote{Identity}‌و همچنین رابط های در دسترس سیستمی که قصد همکاری با آن را دارند اطلاع داشته باشند.در زمان ایجاد همکاری و دریافت پیام از سایر سیستم ها سیستم می تواند یکی از سه رویکرد زیر را در پیش گیرد:
\begin{enumerate}
\item
حالت اول این است که سیستم پاسخ را به سیستمی که درخواست را ارسال نموده است بازگرداند.همانند بسیار از سیستم های \lr{Server-Client}
\item
گاهی ممکن است پاسخ به درخواست کننده ی سرویس بازگشت داده نشود و در عوض پاسخ به سیستمی دیگر ارسال گردد.
\item
گاهی سیستم ما پاسخ را برای تمامی سیستم هایی که ممکن است علاقه مند باشد ارسال می کند.
\end{enumerate}

\section{سناریوی عمومی قابلیت همکاری}
سناریو‌ی عمومی قابلیت همکاری به صورت زیر است:
\begin{itemize}
\item
منبع محرک\LTRfootnote{Source of stimulus} : سیستمی که شروع کننده ی همکاری است.
\item
محرک\LTRfootnote{Stimulus} : در‌خواست تبادیل اطلاعات میان سیستم ها
\item
محصول\LTRfootnote{Artifact}: سیستمی که قصد تبادل اطلاعات با آن وجود دارد.
\item
محیط\LTRfootnote{Enviroment}:مجموعه سیستم هایی که قصد تبادل اطلاعاتی میان آن ها وجود دارد و در هنگام اجرا و یا پیش از آن شناخته می شوند.
\item
پاسخ\LTRfootnote{Response} : درخواست برای همکاری باعث تبادل اطلاعات می شود. اطلاعات توسط طرف گیرنده از نظر نحوی و معنایی قابل درک است.همچنین ، این درخواست رد می شود و نهادهای مربوطه از آن مطلع می شوند. در هر حال ممکن است درخواست های انجام شده توسط سیستم ها ثبت\LTRfootnote{logged} شود. 
\item
اندازه گیری پاسخ\LTRfootnote{Response Measure}: درصدی از اطلاعات که به درستی مبادله شده اند و یا درصدی از اطلاعات مبادله شده که به درستی پذیرفته نشده اند، می توانند روش های اندازه گیری پاسخ باشند.
\end{itemize}

برای دست یابی به هر نیازمندی مجموعه ای از تکنیک ها را به کار می بندیم که در مورد قابلیت همکاری این تکنیک ها شامل مکان یابی\LTRfootnote{Locate}  و مدیریت ارتباط ها \LTRfootnote{Manage Interface} است.

در دسته بندی مکان یابی روش 	کشف سرویس \LTRfootnote{Discover System} در زمان اجرا اقدام به کشف سیستمی که قصد تبادل پیام با آن را داریم می کند.یک سرویس می تواند با استفاده از نوع سرویس،‌نام،‌مکان و سایر خصیصه ها مکان یابی شود.

در مقابل در دسته بندی مدیریت ارتباط ها دو روش معمول هماهنگ سازی\LTRfootnote{Orchestrate} و درخور‌سازی رابط\LTRfootnote{Tailor interface} از جمله روش های معمول به شمار می روند.روش هماهنگ سازی  با استفاده از مکانیزم کنترلی می تواند امکان هماهنگی و مدیریت توالی فراخوانی سرویس های خاص را فراهم سازد و در کنار آن روش درخورسازی ارتباط به افزودن یا کاهش امکانات به یک رابط توجه دارد.

در زمینه ی قابلیت همکاری به دلیل وجود چالش های مشترک در معماری های متفاوت استاندارد های از پیش تعریف شده ی زیادی وجود دارند اما به دلیل سیر تکاملی که استاندارد ها معمولا طی می کنند، نمی توانیم معماری سیستم را بر پایه ی استاندارد ها طراحی کنیم.پیشنهاد می شود که همواره ابتدا با توجه به سایر فاکتور های تاثیر گذار معماری انتخاب شود و سپس از میان استاندارد های منطبق با معماری،‌به انتخاب استاندارد های مناسب با معماری بپردازیم.

\section{طراحی فهرست بازبینی برای قابلیت همکاری}

در ادامه سعی می کنیم تا فهرستی را برای پشتیبانی از روند طراحی و تجزیه و تحلیل برای قابلیت همکاری ارائه کنیم.


\subsection{تخصیص مسئولیت ها} 
% \LTRfootnote{Allocation of Responsibilities}
\subsection{مدل هماهنگی}
% \LTRfootnote{Coordination Model} 
\subsection{مدل داده}
% \LTRfootnote{Data Model} 
\subsection{نقشه برداری در میان عناصر معماری}
% \LTRfootnote{Mapping among Architectural Elements} 
\subsection{مدیریت منابع}
% \LTRfootnote{Resource Management} 
\subsection{زمان اتصال}
% \LTRfootnote{Binding Time}
\subsection{انتخاب فناوری}
% \LTRfootnote{Choice of Technology}








