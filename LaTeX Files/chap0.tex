\chapter{مقدمه}

\section{معرفی صنعت حمل و نقل درون شهری}
برای چندین دهه ، برنامه ریزی و حمل و نقل شهری در اطراف خودرو شخصی تکامل یافته است که منجر به مشکلاتی از جمله ازدحام ، صدا ، آلودگی و غیره شده است.همانطور که وارد دهه جدیدی می شویم، تحول دیجیتال همچنان باعث رشد صنعت می شود. معرفی فن آوری های نوظهور و فرآیندهای تجاری، شیوه لجستیک\LTRfootnote{Logestic} و حمل و نقل را تغییر ساختار داده است و این روند احتمالاً در سال های آتی نیز ادامه خواهد داشت ، خصوصاً که نوآوری در فن آوری منجر به رشد پایدار خواهد شد.

صنعت حمل و نقل امروزه همانند زنجیر صنایع مختلف را بهم متصل می کند؛‌شرکت ها و سازمان های بزرگی بر پایه ی پیشرفت های اخیر صنعت حمل‌و‌نقل فرصت ظهور و خودنمایی پیدا کرده اند و سازمان های بزرگی نظیر Uber در آمریکا، OLA cab در هند و DiDi در چین فراتر از مرز های ملی ظاهر شده و در سراسر جهان شروع به ارائه ی خدمات نموده اند.

در این فصل ابتدا به تاکسی‌رانی آنلاین و سهم آن از حمل‌و‌نقل عمومی از کل صنعت حمل‌و‌نقل عمومی درون شهری خواهیم پرداخت؛سپس روش های اشتراک خودرو\LTRfootnote{CarPool} که به تازگی در حال پیدا کردن جایگاه خود در صنعت حمل‌و‌نقل هستند را مورد بررسی قرار خواهیم داد و در پایان این فصل پیرامون یکی از محیط ترین زیرساخت های شهری یعنی پارکینگ های عمومی و ارتباط آن ها به عنوان یک زیرساخت شهری با صنعت حمل‌و‌نقل عمومی درون شهری صحبت خواهیم کرد.
\section{تاکسی‌رانی و حمل‌و‌نقل درون شهری}
تاکسی‌رانی برخط \LTRfootnote{Ride sharing} امروزه در بسیاری از کشور ها سهم زیادی از بازار را در اختیار خود گرفته اند؛برای مثال Uber تا کنون فعالیت خود را به ۷۰ کشور گسترش داده است و میلیون ها راننده برای این غول بزرگ حمل‌و‌نقل درون شهری فعالیت می کنند و روزانه میلیون ها سفر درون شهری بر بستر این شرکت انجام می شود.

شاید استفاده ی راحت مهم ترین نیازی باشد که شرکت هایی نظیر Uber به آن پاسخ داده اند؛تنها کافی است برنامه ای را بر روی تلفن همراه خود نصب داشته باشید و در کمتر از چند دقیقه نزدیک ترین راننده در حال حرکت به سمت شما برای خدمت رسانی خواهد بود.همچنین راه حل هایی که این شرکت ها در زمینه ی پرداخت هزینه ی سفر،‌تضمین امنیت سفر و کاهش هزینه های سفر ارائه داده اند بر استقبال هر چه بیشتر جامعه از این پلتفرم \LTRfootnote{Platform} های تاکسی برخط\LTRfootnote{Online} افزوده است.

در کنار Uber ، از برنامه های جهانی تاکسی‌رانی برخط می توان به DiDi که عمده فعالیت آن در چین است و هچنین \lr{OLA cab} که در هند فعالیت می کند اشاره کرد.

یکی از موارد مطالعه در این پژوهش پروژه MyWay است.\lr{MyWay} یک پلتفرم یکپارچه، مدیریت منابع در صنعت حمل‌و‌نقل هوشمند اروپا است.هدف این پروژه این است که به طور یکپارچه به ادغام کارآمد و یکپارچه و استفاده از سرویسهای حمل‌و‌نقل مکمل در کل زنجیره سفر شهری بپردازد.

در پایان این بخش لازم است به این نکته اشاره شود که با ورود شرکت های بزرگ تاکسی‌رانی آنلاین به شهر ها، کسب‌و‌کار های حمل‌و‌نقل سنتی آسیب می بینند و همچنین استفاده از وسایل حمل‌و‌نقل عمومی نیز کاهش خواهد یافت که آثار مخرب زیست محیطی و اجتماعی از جمله معایب گسترش این سیستم هاست.

\section{اشتراک خودرو و حمل‌و‌نقل درون شهری}
در سال های اخیر دغدغه های اقتصادی،‌زیست محیطی و اجتماعی جوامع را به سمت استفاده مشترک از منابع در دسترس سوق داده است؛صنعت حمل‌و‌نقل نیز از این قاعده مستثنا نبوده و در جهت استفاده مشترک از زیرساخت های موجود گام‌های موثری را برداشته است.

اشتراک خودرو\LTRfootnote{Car Pooling} به معنی به اشتراک گذاشتن خودرو توسط افراد در سفر هایی با مقاصد نزدیک به هم است و در نتیجه سبب می شود تا افراد کمتری برای رفتن به مقاصد خود از خودرو های شخصی استفاده کنند.

با استفاده بیشتر از یک فرد از وسیله نقلیه، هزینه های سفر هر فرد نظیر  هزینه های سوخت، عوارض رانندگی و استرس رانندگی کاهش خواهد یافت.

رانندگان و مسافران سفرها را از طریق یکی از چندین رسانه موجود ارائه می دهند و جستجو می کنند. آنها پس از یافتن افراد با سفر های مشابه با یکدیگر تماس می گیرند تا جزئیات سفر را ترتیب دهند. هزینه ها،‌نقاط ملاقات و سایر جزئیات مانند فضای وسایل اضافه همراه افراد نیز از پیش توافق شده است.‌‌سپس افراد طبق برنامه ریزی از پیش انجام شده با یکدیگر ملاقات نموده و سفر را انجام می دهند.

هم اکنون استارت آپ های زیادی در زمینه ی اشتراک گذاری خودرو در سرتاسر دنیا شروع به فعالیت کرده اند که می توانیم برای مثال از \lr{Waze Car} ، \lr{BlaBla Car} و \lr{Ride Connect} نام ببریم.

از آنجا که استفاده از اشتراک خودرو تعداد خودر‌و‌های مورد نیاز مسافران را کاهش می دهد ، اغلب با مزایای متعددی در جامعه همراه است از جمله: 
\begin{itemize}
\item
کاهش در مصرف انرژی و انتشار گازهای گلخانه ای
\item
کاهش ترافیک سطح شهری
\item
کاهش تقاضای زیرساخت پارکینگ
\end{itemize}
\section{پارکنیگ های عمومی و حمل‌و‌نقل درون شهری}
برای افرادی که در شهر های بزرگ زندگی می کنند یافتن پارکینگ مقرون به صرفه می تواند یک چالش باشد؛طی تحقیق که توسط IBM انجام شد محققان به این واقعیت پی ردند که نزدیک به  30٪ از ترافیك شهر به رانندگانی كه بدنبال پارکینگ هستند، نسبت داده می شود.خوشبختانه، برنامه های مختلفی برای کمک به افراد در پیدا کردن و مقایسه نقاط پارک مناسب وجود دارند که برخی از آن ها حتی امکان رزرو پارکینگ را در اختیار کاربران خود قرار می دهند.

از جمله برنامه هایی که در زمینه‌ی پارکنیگ های درون شهری در آمریکا فعالیت می کنند می توان به \lr{Best Parking} و یا ParkWhiz اشاره کرد.




















