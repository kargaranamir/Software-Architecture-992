% Chapter 5
\chapter{امنیت}
\section{تعریف امنیت}
امنیت \footnote{Security}
یکی از نیازهای غیر عملکردی سیستم است که بوسیله آن می‌توان توانایی سیستم برای محافظت داده و اطلاعات از اشخاصی که مجوز دسترسی ندارند را نشان داد.
در صورت وجود امنیت در سیستم‌‌های حوزه حمل‌ونقل درون شهری ممکن است که یک دسترسی بدون مجوز بخواهد که به داده کاربران دسترسی پیدا کند و آن را تغییر دهد یا باعث شود که سرویس برای بقیه کاربرهای دارای مجوز در دسترس نباشد.

امنیت ۳ مشخصه اصلی دارد که در ذیل خلاصه شده است \cite{sec_3_princilpe}:
\begin{enumerate}
\item
محرمانگی: در این معیار داده و سرویس‌ها از دسترس دسترسی‌های غیر مجوز دار حفظ می‌شود: برای مثال در سرویس تاکسی آنلاین داده‌های کاربران تنها در اختیار افراد معتبر خواهد بود و هرکسی به آن دسترسی نخواهد داشت.
\item
اصالت: در این معیار اصالت داده‌ها تضمین می‌شود و دست‌خوش تغییرات نمی‌شود: برای مثال در سیستم پارکینگ اگر فضای خالی وجود دارد و نرم‌افزار خالی بودن را نشان می‌دهد این داده اصالت دارد و تنها در صورتی که آن فضا پر شود داده نیز نشان دهنده پر بودن آن است و توسط افراد غیر مجاز دست‌خوش تغییرات نباید بشود.
\item 
در دسترس‌پذیری: در این معیار سیستم برای افراد دارای مجوز در دسترس خواهد بود و برای مثال استفاده از حمله منع سرویس نباید باعث شود که به در دسترس‌پذیری سرویس لطمه وارد شود: برای مثال ممکن است که یک حمله کننده تعداد زیادی درخواست به سرویس اشتراک خودرو دهد و سرویس آن بدلیل بار زیاد از دسترس کاربران دیگر خارج شود.
\end{enumerate}

همچنین مشخصه‌های دیگری را نیز می‌توان برای امنیت لحاظ کرد که زیر مجموعه‌ معیارهای قبلی هستند. 
در ذیل به ۳ مورد از آن‌ها اشاره شده است.
\begin{enumerate}
\item
احراز هویت: هویت افرادی که با سیستم درگیر هستند را تایید می‌کند: برای مثال در سیستم تاکسی آنلاین با استفاده از نام کاربری و رمز عبور و شماره همراه به تایید کاربر پرداخته می‌شود.
\item
عدم پذیرش مسئولیت: 
تضمین می‌کند که فرستنده در آینده فرستادن پیام خود را انکار نمی‌کند: برای مثال در سامانه اشتراک خودرو بین افرادی که یک خودرو را به اشتراک می‌گذارند عدم پذیرش مسئولیت پیام‌های رد و بدل شده بوجود نیاید که این موضوع ممکن است در آینده باعث غش در توافق بین‌ آن‌ها بینجامد.
\item
احراز اصالت: به کاربر مجوزهای لازم برای انجام وظیفه مربوطه را اعطا می‌کند: برای مثال در سامانه پارکینگ افراد احراز هویت شده مجوز لازم برای رزرو جای پارک را داشته باشند.
\end{enumerate}

\section{سناریوی عمومی امنیت}
سناریو‌ی عمومی امنیت به صورت زیر است:
\begin{itemize}
\item
منبع محرک\LTRfootnote{Source of stimulus} : انسان یا سیستمی که تشخیص داده شده است حال چه به صورت صحیح چه ناصحیح. برای مثال یک حمله‌کننده انسان خارج از شرکت سرویس تاکسی آنلاین
\item
محرک\LTRfootnote{Stimulus} : یک در‌خواست بدون مجوز برای نمایش داده، تغییر، حذف یا دسترسی به سرویس ها: یک درخواست بدون مجوز توسط یک فرد تایید نشده برای حذف رانندگان مجاز از سامانه اشتراک خودرو
\item
محصول\LTRfootnote{Artifact}: 
سرویس‌های سیستم، داده‌های درون سیستم، مولفه‌ها و منابع سیستم: برای مثال تمام منابع موجود در سیستم‌های نرم‌افزاری حوزه حمل‌ و نقل درون شهری
\item
محیط\LTRfootnote{Enviroment}:
سیستم که می‌تواند به صورت برون خط یا برخط باشد، به شبکه متصل باشد یا نباشد، پشت دیواره آتش باشد یا نباشد: برای مثال سیستم‌های برخط حوزه حمل و نقل درون شهری که پشت دیواره آتش هستند 
\item
پاسخ\LTRfootnote{Response} : بسته به نوع محرک و محصول پاسخ‌هایی بدین شکل تولید خواهد شد: داده و سرویس از دسترسی بدون مجوز محافظت شود، تغییر نیابد. منابع و سرویس‌ها در دسترس باشد و ... .همچنین تمامی فعالیت‌ها ضبط شود تا بعدا بتوان در صورت بروز خرابی علت را پیگیری کرد. 
\item
اندازه گیری پاسخ\LTRfootnote{Response Measure}: می‌توان معیارهای متفاوتی در نظر گرفت: چه مقدار طول کشیده است تا سیستم بعد از یک حمله موفق دوباره بازیابی شود، چه مقدار از مولفه‌های سیستم یا داده مورد خرابی قرار گرفته است یا چقدر طول کشیده است تا متوجه حمله شود.
\end{itemize}

برای دست یابی به هر نیازمندی مجموعه ای از تکنیک ها را به کار می بندیم که در مورد امنیت می‌توان به ۴ دسته شناخت، مقاومت، عکس‌العمل و بازیابی اشاره کرد.


برای شناخت حمله می‌توان از مقایسه کردن ترافیک شبکه یا سرویس یا مقایسه پترن‌های سیستم و پیدا کردن رفتار‌های غیر معمول بهره برد.
همچنین از روی چنین پترن‌هایی می‌توان به حمله منع سرویس نیز پی برد یا از روی دیرکرد پیام به آن مشکوک شد. برای چک کردن اصالت داد‌ه‌ها نیز می‌توان از روش‌های چک کردن اصالت داده مانند کدهای خطا\footnote{Checksum}
و توابع هش\footnote{Hash}استفاده کرد.

به منظور مقاومت می‌توان منابع ورودی به سیستم را شناسایی کرد و مطمئن شد که آن منبع دقیقا همان کس یا چیزی است که باید باشد. به منابع مطمئن مجوزهای لازم برای تغییر و دسترسی به داده اعطا شود و تا حد ممکن دسترسی به منابع محدود شود. 
همچنین منابع را از نظر فیزیکی تا جای ممکن ایزوله باشد تا اگر یکی مورد حمله قرار گرفت دیگری را تحت تاثیر قرار ندهد. همچنین می‌تواند از مندهای رمزنگاری بر روی داده و ارتباط استفاده کرد و کاربر را مجبور کرد که تنظیمات خود را از حالت معمول خارج کرده و آن را تغییر دهد.

عکس العمل مناسب در برابر حمله شامل اقدامات باطل کردن دسترسی حمله به منابع حتی برای کاربرهای مجاز است تا زمانی که حمله رفع شود. 
اگر تلاش‌ها مکرری برای دسترسی موفقیت آمیز نبود، دسترسی محدود و قفل شود. به اپراتورها و افراد مناسب در زمان اتفاق افتادن امری مشکوک یا تشخیص حمله اطلاع داده شود.

به منظور بازیابی مناسب از تاکتیک‌های مربوط به دسترس‌پذیری استاده شود تا  منابع بازیابی شوند و از تمامی اتفاقات رکورد تهیه شود تا بعدا بتوان حمله کننده را تشخیص داد و آن را ردیابی کرد.

به منظور بهره بری از این تاکتیک‌ها می‌توان با توجه به هریک از پروژه‌های تاکسی آنلاین، اشتراک‌گذاری خودرو و مدیریت پارکینگ یک فهرست بازبینی تنظیم کرد. این موضوع به بهبود نیاز امنیت پاسخ می‌دهد.

\section{طراحی فهرست بازبینی برای امنیت}

در ادامه فهرستی برای پشتیبانی از روند طراحی و تجزیه و تحلیل برای معیار امنیت ارائه شده است.


\subsection{تخصیص مسئولیت ها} 
در این گام نیاز است که تعیین شود چه مسئولیت‌هایی از سیستم نیاز است که امن شود. برای این مسئولیت‌ها نیاز است که مطمئن شد مسئولیت‌های اضافه تخصیص یافته است:
\begin{itemize}
\item
منبع محرک شناسایی شود.
\item
منبع محرک احراز اصالت شود.
\item
منبع محرک احراز هویت شود.
\item
مجوز‌های لازم برای دسترسی به داده و سرویس‌ها اعطا شود.
\item
داده‌ها رمزگذاری شود.
\item 
اگر دردسترس‌پذیری کاهش یافت تشخیص داده شود و اقدام مناسب مانند محدود کردن دسترسی و مطلع کردن اپراتور صورت گیرد.
\item
بعد از حمله بازیابی صورت گیرد.
\item
کدهای خطا و مقادیر هش تایید شود.
\end{itemize}
\subsection{مدل هماهنگی}
در این گام نیاز است که مکانیزم‌های لازم برای ارتباط با بخش‌های دیگر تعیین شود. همچنین باید مطمئن شد که این این مکانیزم‌ها از قابلیت احراز هویت و اصالت بهره می‌برند و داده به صورت رمزشده میان بخش‌های مختلف انتقال پیدا می‌کند.
در نهایت باید مطمئن شد که مکانیزم‌هایی برای مانیتور کردن و تشخیص تقاضا منابع زیاد وجود دارد و در صورت بروز تقاضا آن‌ها را محدود کرد.

\subsection{مدل داده}
در این بخش حساسیت فیلد‌های مختلف داده تعیین شود و موارد زیر اجرا گردد:
\begin{itemize}
\item
باید مطمئن شد که داده‌های با سطح حساسیت متفاوت به صورت جدا ذخیره شده است.
\item
باید مطمئن شد که برای دسترسی به داده‌های با سطح حساسیت متفاوت، دسترسی‌هایی با سطح متفاوت نیاز است.
\item
باید مطمئن شد که درخواست دسترسی به داده‌های حساس ضبط می‌شود و از داده‌های ضبط آن نیز به طور مناسب محافظت می‌شود.
\item 
باید مطمئن شد که داده‌ها به شکل مناسب محافظت شده است و کلید‌های آن نیز جدا از داده‌ها  ذخیره و محافظت می‌شوند.
\item
باید مطمئن شد که اگر داده ‌ها به طور نامناسب تغییر پیدا کرد قابل بازیابی باشد.
\end{itemize}

\subsection{نگاشت در میان عناصر معماری}
در این قسمت تعیین می‌شود که چگونه نگاشت جایگزین اجزا معماری ممکن است که نحوه خواندن و نوشتن و تغییر دادگان و تنظیم دسترسی‌ها و کاهش دردسترس پذیری را تحت تاثیر قرار دهد. همچنین 
تعیین می‌شود که چگونه نقشه‌برداری ضبط داده‌های دسترسی به دادگان و سرویس‌ها را تحت تاثیر قرار می‌دهد یا تقاضاهای زیاد سیستم را تشخیص می‌دهد.  
برای هر یک از این نقشه برداری‌ها باید مطمئن شد که مسئولیت‌های عنوان شده در بخش تخصیص مسئولیت وجود دارد.
\subsection{مدیریت منابع}
در این بخش نیاز است که منابعی از سیستم که نیاز است شناسایی، مانیتور، احراز اصالت و دسترسی خاص به منابع داشته باشند تعیین شود.
همچنین نیاز است که منابعی که نیاز است تا احراز هویت، دادن مجوز، اطلاع افراد مناسب، رکورد دسترسی به داده‌ها، رمزگذاری داده‌ها و تشخیص تقاضای زیاد و محدود کردن دسترسی داشته باشند تعیین شود.
برای این منابع نیاز است که تعیین شود که افراد خاص سازمان چقدر به منابع مخصوصا منابع حساس دسترسی داشته باشند. چگونه منابع مانیتور شوند و مطمئن شد که بخش‌های لازم عملیات‌های امنیتی مورد نیاز را انجام می‌دهند. 
همچنین باید مطمئن شد که از منابع مشترک برای انتقال داده‌های حساس از یک کاربری با حق دسترسی به یک کاربری بدون حق دسترسی استفاده نمی‌شود.

\subsection{زمان اتصال}
مواردی را در نظر بگیرید که یک مولفه دیررس قابل اعتماد نباشد. 
به این منظور باید از مکانیزم‌های مناسب برای مدیریت و اعتبارسنجی آن‌ استفاده کرد. دسترسی آن مولفه می‌تواند به سرویس‌ها و داده‌ها مسدود شود و دسترسی‌های ضبط شود. همچنین دادگان نیز رمزگذاری و کلید آن‌ها در جایی که اجزا دیررس دسترسی ندارند ذخیره شود. تمام این موارد انجام شود تا مولفه ارزیابی شود و صلاحیت آن اثبات شود.

\subsection{انتخاب فناوری}
از فناوری‌هایی که می‌توان بوسیله آن از داد‌ها محافظت نمود و حق دسترسی‌ها را به خوبی مدیریت کرد و قابلیت احراز هویت کاربران را فراهم نمود باید استفاده کرد.
همچنین باید دقت داشت که تکنولوژی مورد انتخاب تاکتیک‌های مورد نیاز برای بحث امنیت را پشتیبانی کند.


\section{مطالعات موردی}
در اینجا مطالعات موردی در ۳ حوزه حمل و نقل درون شهری برای نیاز امنیت انجام شده است.
همانگونه که قبلا بیان شد باید بسته یه قسمت‌های حساس پروژه از رمزگذاری داد‌ه‌ها و ارتباط، شناسایی، احراز هویت و اصالت، کاهش دسترسی‌پذیری، بازیابی، ذخیره مناسب کلید‌ها و کدهای هش و خطا استفاده کرد تا امنیت مورد نیاز تامین شود.
 
در مستندات پروژه MyWay اعلام شده است که اطلاعات کاربر ناشناس می‌ماند، پروفایل‌های کاربر در سرور ذخیره می‌شود و حریم شخصی و مدیریت مناسب انجام می‌شود. ذخیره پروفایل‌های کاربر با قوانین محلی سازگار است، ارتباطات برای داده‌های حساس به صورت رمز شده است.  قابلیت ردیابی برای عملیات‌های حساس انجام شده توسط کاربران وجود دارد، به خصوص برای ادمین‌ها. همچنین این اپلیکیشن صحت داده و مجوز‌های لازم برای دسترسی به داده را تضمین می‌کند. ادمین‌ها دسترسی لازم برای مشاهده و تغییر داده‌های درونی را نیز دارند.
در آخر باید گفت که کاربران در این اپلکیشن احراز هویت می‌شود و سطح دسترسی‌ها و مجوزهای مختلفی میان نقش‌های اپلیکیشن وجود دارد. کاربر احراز هویت شده می‌تواند به نقش‌های مختلفی نگاشت شود \cite{myway_req}.
همچنین با بررسی‌ای که از برنامه Uber ، OLA ، ParkWhiz و Waze-Carpool به عمل آمد مشخص شد که این اپلیکیشن‌ها از قابلیت‌های احراز هویت با استفاده از نام کاربری و رمز عبور، نگه داشتن نشست کاربر و دنبال کردن فعالیت‌های کاربر، ضبط رکوردهای درخواست کاربر به سرور، استفاده از متد‌های رمزنگاری مناسب، نگه‌داشتن تاریخچه مناسب از دادگان، نسبت دادن برخی توابع خاص تنها به بعضی از ماژول‌ها، محدود کردن ارتباط تنها در قسمت‌هایی از برنامه،‌ چک کردن صحت دادگان برای متغیرهای حیاتی، دنبال کردن خطاها، استفاده از ارتباط امن ssl  برخوردار هستند.
