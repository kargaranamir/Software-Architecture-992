% Chapter 1
\chapter{دسترسی‌پذیری}
\section{تعریف دسترسی پذیری}
دسترسی پذیری\LTRfootnote{Availability} به خاصیتی از نرم افزار اطلاق می شود که نرم افزار در لحظه حاضر باشد و از آمادگی انجام وظایف خود برخوردار باشد.در واقع دسترسی پذیری بر پایه ی مفهوم قابلیت اطمینان\LTRfootnote{Reliability} بنا شده با این تفاوت که مفاهیم ترمیم \LTRfootnote{Recovery} را نیز به آن می افزاید.

به طور دقیق تر،‌قابلیت دسترس‌پذیری به توانایی سیستم برای پنهان کردن یا اصلاح شکست ها اشاره دارد؛‌به طوری که مجموع مدت زمان قطع خدمات از یک مقدار لازم در یک بازه زمانی مشخص بیشتر نشود.شکست از زاویه دید سیستم یا ناظر انسانی در محیط تعریف می شود و به معنای انحراف سیستم از مشخصات\LTRfootnote{Specification} آن است به گونه ای که این انحراف قابل مشاهده باشد.آنچه به شکست منجر می شود،خطاست و معمار نرم افزار همواره سعی می کند تا با درک ریشه های شکست سیستمی مقاوت در برابر خطا\LTRfootnote{Resilient} را بنا کند.

از جمله مواردی که دغدغه ی معماری سیستم در زمینه ی دسترسی پذیری است می توان به موارد زیر اشاره کرد:
\begin{itemize}
\item
چگونه خطا های سیستم باید شناسایی شوند؟
\item
چقدر ممکن است خطا های سیستم رخ دهند و عواقب رخداد یک خطا چیست؟
\item
یک سیستم چقدر می تواند خارج از دسترس باشد؟
\item
چگونه می توان از خطا و در نتیجه شکست نرم افزار جلوگیری کرد؟
\item
در زمان رخداد خطا چه اقداماتی لازم هستند؟
\end{itemize}