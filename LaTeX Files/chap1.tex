% Chapter 1
\chapter{دسترسی‌پذیری}
\section{تعریف دسترسی پذیری}
دسترسی پذیری\LTRfootnote{Availability} به خاصیتی از نرم افزار اطلاق می شود که نرم افزار در لحظه حاضر باشد و از آمادگی انجام وظایف خود برخوردار باشد.در واقع دسترسی پذیری بر پایه ی مفهوم قابلیت اطمینان\LTRfootnote{Reliability} بنا شده با این تفاوت که مفاهیم ترمیم \LTRfootnote{Recovery} را نیز به آن می افزاید.

به طور دقیق تر،‌قابلیت دسترس‌پذیری به توانایی سیستم برای پنهان کردن یا اصلاح شکست ها اشاره دارد؛‌به طوری که مجموع مدت زمان قطع خدمات از یک مقدار لازم در یک بازه زمانی مشخص بیشتر نشود.شکست از زاویه دید سیستم یا ناظر انسانی در محیط تعریف می شود و به معنای انحراف سیستم از مشخصات\LTRfootnote{Specification} آن است به گونه ای که این انحراف قابل مشاهده باشد.آنچه به شکست منجر می شود،خطاست و معمار نرم افزار همواره سعی می کند تا با درک ریشه های شکست سیستمی مقاوت در برابر خطا\LTRfootnote{Resilient} را بنا کند.

از جمله مواردی که دغدغه ی معماری سیستم در زمینه ی دسترسی پذیری است می توان به موارد زیر اشاره کرد:
\begin{itemize}
\item
چگونه خطا های سیستم باید شناسایی شوند؟
\item
چقدر ممکن است خطا های سیستم رخ دهند و عواقب رخداد یک خطا چیست؟
\item
یک سیستم چقدر می تواند خارج از دسترس باشد؟
\item
چگونه می توان از خطا و در نتیجه شکست نرم افزار جلوگیری کرد؟
\item
در زمان رخداد خطا چه اقداماتی لازم هستند؟
\end{itemize}


\section{سناریوی عمومی دسترسی‌پذیری}
سناریو‌ی عمومی دسترسی‌پذیری به صورت زیر است:
\begin{itemize}
\item
منبع محرک\LTRfootnote{Source of stimulus} : چه درون و چه بیرون: افراد، سخت‌افزار، نرم‌افزار، محیط فیزیکی 
\item
محرک\LTRfootnote{Stimulus} : خطا: حذف، خرابی، زمان بندی نادرست، پاسخ نادرست
\item
محصول\LTRfootnote{Artifact}: پروسسورهای سیستم، کانال‌های ارتباطی، ذخیره‌سازی مداوم، فرآیند‌ها
\item
محیط\LTRfootnote{Enviroment}: عملکرد عادی، راه اندازی، خاموش کردن، حالت تعمیر، عملکرد تخریب شده، عملکرد بیش از حد
\item
پاسخ\LTRfootnote{Response} : جلوگیری از این که یک خطا منجر به یک شکست شود. به این منظور نیاز است که ابتدا خطا شناخته شود. در صورت وجود خطا تاریخچه آن حفظ شود و سیستم‌ها و آدم‌های مرتبط اطلاع داده شود. همچنین نیاز است که بازیابی پس از انجام خطا نیز انجام شود که شامل غیرفعال کردن منبعی که باعث خطا شده است، غیرفعال کردن تا زمانی که تعمیر تکمیل شود، تعمیر خطا و شکست و عمل کردن در مد کاهش‌داده شده\footnote{\lr{degraded mode}} تا زمانی که تعمیر تکمیل شود.
\item
اندازه گیری پاسخ\LTRfootnote{Response Measure}: موارد بسیاری را می‌توان در نظر گرفت که به مقدار زمانی که سیستم دردسترس است اشاره دارد. از جمله آن‌ها می‌توان به درصد در دسترسی‌پذیری، زمانی که تا تشخیص خطا طول می‌کشد، زمانی که تا تعمیر طول ‌می‌کشد، مقدار زمانی که سیستم در حالت کاهش یافته عمل می‌کند اشاره کرد. 
\end{itemize}

\section{تاکتیک ها در دسترسی‌پذیری}
تاکتیک های در دسترس‌پذیری به گونه ای طراحی شده اند که سیستم را قادر می سازد تا خطاهای سیستم را تحمل کند تا سرویس ارائه شده توسط سیستم با مشخصات آن مطابقت داشته باشد و به شکست ختم نشود.تاکتیک های دسترسی‌پذیری را می توان به سه دسته ی تشخیص خطا\lr{fault detection} ،بازیابی از خطا\lr{fault recovery} و جلوگیری از خطا\lr{fault prevention} تقسیم بندی کرد.
\subsection{تشخیص خطا}
اولین قدم برای جلوگیری از شکست تشخیص خطاست.به منظور تشخیص خطا سیستم می تواند از روش های زیر استفاده کند:
\begin{itemize}
\item
Ping / Echo
\item
نظارت\LTRfootnote{Monitor}
\item
Heartbeat
\item
Timestamp
\item
عقل سنجی\LRTfootnote{Sanity Checking}
\item
نظارت بر شرایط\LTRfootnote{Condition Monitoring}
\end{itemize}
\subsection{بازیابی از خطا}

\subsection{جلوگیری از خطا}

\section{طراحی فهرست بازبینی برای دسترسی‌پذیری}















