% Chapter 1
\chapter{دسترسی‌پذیری}
\section{تعریف دسترسی پذیری}
دسترسی پذیری\LTRfootnote{Availability} به خاصیتی از نرم افزار اطلاق می‌شود که نرم افزار در لحظه، حاضر باشد و از آمادگی انجام وظایف خود برخوردار باشد.
در واقع دسترسی پذیری بر پایه ی مفهوم قابلیت اطمینان\LTRfootnote{Reliability} بنا شده با این تفاوت که مفاهیم ترمیم \LTRfootnote{Recovery} نیز به آن افزوده شده است.

به طور دقیق‌تر، قابلیت دسترس‌پذیری به توانایی سیستم برای پنهان کردن یا اصلاح شکست‌ها اشاره دارد؛‌ به طوری که مجموع مدت زمان قطع خدمات از یک مقدار لازم در یک بازه زمانی مشخص بیشتر نشود. شکست از زاویه دید سیستم یا ناظر انسانی در محیط تعریف می شود و به معنای انحراف سیستم از مشخصات\LTRfootnote{Specification} آن است به گونه ای که این انحراف قابل مشاهده باشد. آنچه به شکست منجر می شود، خطاست و معمار نرم افزار همواره سعی می کند تا با درک ریشه های شکست سیستمی مقاوت در برابر خطا\LTRfootnote{Resilient} را بنا کند.

از جمله مواردی که دغدغه ی معماری سیستم در زمینه ی دسترسی‌پذیری است می توان به موارد زیر اشاره کرد:
\begin{itemize}
\item
چگونه خطا‌های سیستم باید شناسایی شوند؟
\item
چقدر ممکن است خطا‌های سیستم رخ دهند و عواقب رخداد یک خطا چیست؟
\item
یک سیستم چقدر می تواند خارج از دسترس باشد؟
\item
چگونه می توان از خطا و در نتیجه شکست نرم افزار جلوگیری کرد؟
\item
در زمان رخداد خطا چه اقداماتی لازم هستند؟
\end{itemize}


\section{سناریوی عمومی دسترسی‌پذیری}
سناریو‌ی عمومی دسترسی‌پذیری به صورت زیر است:
\begin{itemize}
\item
منبع محرک\LTRfootnote{Source of stimulus} : چه درون و چه بیرون: افراد، سخت‌افزار، نرم‌افزار، محیط فیزیکی 
\item
محرک\LTRfootnote{Stimulus} : خطا: حذف، خرابی، زمان بندی نادرست، پاسخ نادرست
\item
محصول\LTRfootnote{Artifact}: پروسسورهای سیستم، کانال‌های ارتباطی، ذخیره‌سازی مداوم، فرآیند‌ها
\item
محیط\LTRfootnote{Enviroment}: عملکرد عادی، راه اندازی، خاموش کردن، حالت تعمیر، عملکرد تخریب شده، عملکرد بیش از حد
\item
پاسخ\LTRfootnote{Response} : جلوگیری از این که یک خطا منجر به یک شکست شود. به این منظور نیاز است که ابتدا خطا شناخته شود. در صورت وجود خطا تاریخچه آن حفظ شود و سیستم‌ها و آدم‌های مرتبط اطلاع داده شود. همچنین نیاز است که بازیابی پس از انجام خطا نیز انجام شود که شامل غیرفعال کردن منبعی که باعث خطا شده است، غیرفعال کردن تا زمانی که تعمیر تکمیل شود، تعمیر خطا و شکست و عمل کردن در مد کاهش‌داده شده\footnote{\lr{degraded mode}} تا زمانی که تعمیر تکمیل شود.
\item
اندازه گیری پاسخ\LTRfootnote{Response Measure}: موارد بسیاری را می‌توان در نظر گرفت که به مقدار زمانی که سیستم دردسترس است اشاره دارد. از جمله آن‌ها می‌توان به درصد در دسترسی‌پذیری، زمانی که تا تشخیص خطا طول می‌کشد، زمانی که تا تعمیر طول ‌می‌کشد، مقدار زمانی که سیستم در حالت کاهش یافته عمل می‌کند اشاره کرد. 
\end{itemize}

\section{تاکتیک ها در دسترسی‌پذیری}
تاکتیک‌ها در دسترس‌پذیری به گونه ای طراحی شده است که سیستم را قادر می‌سازد تا خطاهای سیستم را تحمل کند تا سرویس ارائه شده توسط سیستم با مشخصات آن مطابقت داشته باشد و به شکست ختم نشود. تاکتیک های دسترسی‌پذیری را می توان به سه دسته‌ی تشخیص خطا\lr{fault detection} ،بازیابی از خطا\lr{fault recovery} و جلوگیری از خطا\lr{fault prevention} تقسیم بندی کرد.
\subsection{تشخیص خطا}
اولین قدم برای جلوگیری از شکست تشخیص خطاست.به منظور تشخیص خطا سیستم می تواند از روش های
پینگ \footnote{Ping}، اکو \footnote{Echo} ،
نظارت\LTRfootnote{Monitor}
،Heartbeat
،Timestamp
عقل سنجی\LTRfootnote{Sanity Checking}،
نظارت بر شرایط\LTRfootnote{Condition Monitoring}،
رای دادن \LTRfootnote{Voting}،
شناسایی استثنا\LTRfootnote{Exception Detection}،
خودآزمونی \LTRfootnote{self-test}
استفاده کرد.
\subsection{بازیابی از خطا}
تاکتیک‌های بازیابی از خطاها در قالب روش‌های آماده‌سازی و ترمیم و تکنیک‌های بازآفرینی سیستم پیاده سازی می شوند. روش‌های بازآفرینی دغدغه‌ی راه اندازی مجدد یک سیستم شکست خورده را دارند. روش‌های ترمیم 
افزونگی فعال \LTRfootnote{Active redundancy}،
افزونگی منفعل \LTRfootnote{Passive redundancy}،
پشتیبان \LTRfootnote{Spare}،
کنترل استثنا \LTRfootnote{Exception handling}،
بازگشت به عقب \LTRfootnote{Rollback}،
به روز‌رسانی نرم‌افزار \LTRfootnote{Software Upgrade}،
امتحان مجدد \LTRfootnote{Try}،
نادیده گرفتن رفتار دارای خطا،
تخریب \LTRfootnote{The degradation}،
پیکربندی ،\LTRfootnote{Reconfiguration}
می‌باشند.

و روش های بازآفرینی سیستمی که پس از شکست از اصلاح شده است، شامل
سایه \LTRfootnote{The Shadow}،
هماهنگ سازی مجدد \LTRfootnote{State resynchronization}،
راه اندازی مجدد افزایشی \LTRfootnote{Escalating restart}
است.

\subsection{جلوگیری از خطا}
به جای شناسایی خطاها و سپس تلاش برای بازیابی آن‌ها، در این تاکتیک سیستم سعی می کند تا از خطا جلوگیری کند .هر چند انجام چنین کاری در نگاه اول دشوار بنظر می رسد اما در بسیاری از موارد انجام چنین کاری ممکن است.
از روش‌های
حذف از سرویس \LTRfootnote{Removal from service}،
تراکنش ها \LTRfootnote{Transactions}،
مدل پیش‌بینی \LTRfootnote{Predictive Model}،
پیشگیری از استثنا \LTRfootnote{Exception prevention}،
افزایش مجموعه‌ی صلاحیت‌ها \LTRfootnote{Increase competence set}
به این منظور می‌توان استفاده کرد.
\section{طراحی فهرست بازبینی برای دسترسی‌پذیری}
\subsection{تخصیص مسئولیت ها} 
در زمان تخصیص مسئولیت ها باید \LTRfootnote{Allocation of Responsibilities} به موارد زیر توجه شود:
\begin{itemize}
\item
مسئولیت های سیستم را که باید بسیار در‌دسترس باشند
\item
اطمینان حاصل شود که مسئولیت‌های اضافی برای تشخیص حذف، خرابی، زمان‌بندی نادرست یا پاسخ نادرست اختصاص داده شده است.
\item
اطمینان حاصل شود وظائف زیر به بخش‌هایی از سیستم تخصیص داده شده است:
\begin{itemize}
\item
ثبت\LTRfootnote{Log} خطا‌ها
\item
اطلاع‌رسانی به موجودیت های موجود در سیستم
\item
غیر‌فعال‌سازی منشا خطا
\item
خارج شدن از دسترس به صورت موقت
\item
اصلاح یا پوشاندن\LTRfootnote{mask} خطا
\item
فعالیت سیستم به حالت کاهش‌یافته \LTRfootnote{Degraded} برده شود
\end{itemize}
\end{itemize}

\subsection{مدل هماهنگی}
پس از تعیین مهم‌ترین قسمت‌های سیستم که باید بسیار در‌دسترس باشند، می بایست موارد زیر در نظر گرفته شوند:
\begin{itemize}
\item
اطمینان حاصل شود که مکانیسم‌های هماهنگی می توانند حذف، خرابی، زمان‌بندی نادرست یا پاسخ نادرست را تشخیص دهند. برای مثال در نظر بگیرید که آیا هماهنگی نرم‌افزار تاکسی آنلاین در شرایط ارتباطات ضعیف کار خواهد کرد؟
\item
اطمینان حاصل شود که مکانیزم های هماهنگی امکان ثبت گزارش از خطا، اطلاع رسانی به موجودیت های مناسب، غیرفعال کردن منبع ایجاد کننده خطا، رفع یا پوشاندن خطا و یا عملکرد در حالت کاهش یافته را دارند.
\item
اطمینان حاصل شود که مدل هماهنگی از جایگزینی مصنوعات استفاده شده مثل پردازنده ها، کانال های ارتباطی، سخت افزار های ذخیره سازی و فرآیندها پشتیبانی می کند. به عنوان مثال، آیا جایگزینی سرور در یک سیستم پارکینگ عمومی اجازه می دهد سیستم به کار خود ادامه دهد؟
\item
تعیین کنید که آیا این هماهنگی در شرایط ارتباطات کاهش یافته، هنگام راه اندازی\LTRfootnote{Startup}/ خاموش شدن\LTRfootnote{Shutdown}، در حالت تعمیر\LTRfootnote{Repair mode} یا تحت کار بیش از حد\LTRfootnote{OverLoaded} به درستی فعالیت می کند. به عنوان مثال، مدل هماهنگی در نرم‌افزار اشتراک خودرو چقدر اطلاعات از دست رفته\LTRfootnote{Lost information} را تحمل می کند و با چه عواقبی؟
\end{itemize}

\subsection{مدل داده}
لازم است ابتدا مشخص شود که کدام قسمت از سیستم باید همواره در دسترس باشد سپس، در این قسمت ها، تعیین شود که انتزاع داده ها، همراه با عملیات\LTRfootnote{Operation} هایی که برای آن ها تعریف شده اند، می توانند باعث کدوم یک از خطاهای حذف، خرابی\LTRfootnote{crash}، رفتار نادرست در زمان بندی یا پاسخ نادرست شوند.
همچنین برای این داده‌های انتزاعی و عملکردها و خصوصیات آن‌ها، اطمینان حاصل شود که در صورت لزوم می توانند به طور موقت غیرفعال شوند؛ به عنوان مثال، اطمینان حاصل شود زمانی که سرور مدیریت پارکینگ عمومی موقتا در دسترس نیست، درخواست های نوشتن بر روی دیسک در حافظه ی نهان ذخیره می شوند تا پس از بازگشت مجدد سرور، پردازش فرآیند آن‌ها ادامه پیدا کند.
\subsection{نگاشت در میان عناصر معماری}
مشخص شود کدام بخش از سیستم نظیر پردازنده‌ها، کانال‌های ارتباطی و یا سایر بخش‌ها می تواند منجر به ایجاد خطا در سیستم شود. همچنین اطمینان حاصل شود نگاشت میان عناصر معماری از انعطاف کافی جهت بهبود\LTRfootnote{Recovery} پس از شکست برخوردار باشد.
\subsection{مدیریت منابع}
در این بخش باید مشخص شود چه منابع حیاتی برای ادامه کار در صورت وجود خطا لازم است و باید از وجود منابع کافی برای ثبت\LTRfootnote{Log} خطا، اطلاع‌رسانی خطا به سایر موجودیت‌ها، غیر‌فعال‌سازی منشا خطا و تعمیر و بازیابی سیستم از آن، اطمینان حاصل شود.

زمان در‌دسترس بودن منابع حیاتی مشخص شود و تعیین شود چه منابعی باید در شرایط خاص همچون زمانی که سیستم با خطا مواجه شده و یا در حالت کاهش یافته در حال فعالیت است، باید در‌دسترس باشند. به عنوان مثال، اطمینان حاصل کنید که صفهای ورودی در سیستم اشتراک خودرو به اندازه کافی بزرگ هستند تا در صورت خرابی سرور پیامهای پیش بینی شده را بافر\LTRfootnote{buffer} کنند تا پیامها برای همیشه از بین نرود.
\subsection{زمان اتصال}
نحوه و زمان اتصال عناصر معماری مشخص شود. اگر از اتصال دیرهنگام برای جایگزینی بین اجزایی استفاده شود که خود می توانند منبع خطا در سیستم باشد، اطمینان حاصل شود که استراتژی در دسترس بودن تعیین شده برای پوشش خطاهای احتمالی تولید‌شده توسط همه‌ی منابع کافی است. به عنوان مثال، اگر از اتصال دیرهنگام در سیستم تاکسی آنلاین برای جابجایی بین قطعاتی مانند پردازنده هایی که ممکن است در معرض خطا قرار گیرند استفاده شود ، آیا مکانیزم های تشخیص و بازیابی خطا تعیین شده برای همه اتصال های احتمالی به درستی عمل می کند؟
\subsection{انتخاب فناوری}
فناوری های موجود به گونه‌ای تعیین شود که بتوانند به شناسایی خطا‌ها، بازیابی از آن‌ها و یا بازآفرینی مجدد بخش هایی از سیستم که با خطا متوقف شده اند، کمک کند.

همچنین ویژگیهای در دسترس بودن خود فناوری‌های انتخاب شده تعیین شود: پس از برخورد با چه خطا‌هایی قادر به بازیابی مجدد خواهند بود؟ چه نقص هایی ممکن است توسط این فناوری ها در سیستم وارد شود؟


\section{مطالعات موردی}

در اینجا مطالعات موردی در ۳ حوزه حمل و نقل درون شهری برای نیاز دسترس‌پذیری انجام شده است. 
\subsection{تاکسیرانی و حمل‌و‌نقل درون شهری}

اپلیکیشن MyWay : 
اپلیکیشن OLA : تمام دیتای کش شده باید در هر راه‌اندازی دوباره ساخته شود. داده کاربر در صورتی که گم شود، متدی برای ریکاوری آن وجود نخواهد داشت. قابلیت بازگشت مقادیر سیستم به حالت پیش‌فرض اگر لازم باشد.

\subsection{اشتراک خودرو و حمل‌و‌نقل درون شهری}


\subsection{پارکنیگ های عمومی و حمل‌و‌نقل درون شهری}

















