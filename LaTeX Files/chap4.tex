% Chapter 2
\chapter{کارایی}
\section{تعریف کارایی}
کارایی\LTRfootnote{Performance} به توانایی سیستم در پاسخ گویی به نیازها در زمانی مشخص اشاره دارد. زمانی که رخدادی جدید نظیر خواسته ای از جانب کاربر یا سایر سیستم‌ها پیش می آید سیستم باید در بازه ی زمانی مورد انتظار آن رخداد را بررسی کند و به نیاز آن پاسخ دهد. در عموم سیستم‌های حوزه حمل و نقل که به صورت برخط و مبتنی بر وب هستند ممکن است کارایی با گزاره هایی نظیر تعداد درخواست بر دقیقه بیان شود.

کارایی در بیشتر اوقات درکنار گسترش پذیری\LTRfootnote{Scalability} شنیده می شود،‌گسترش پذیری به افزایش ظرفیت پاسخ دهی با حفظ کارایی اشاره دارد و نباید مفهوم آن با کارایی یکسان در نظر گرفته شود.
\section{سناریو عمومی کارایی}
سناریو کارایی با فرارسیدن رویداد‌ها به سیستم آغاز می شود. سیستم باید هم زمان با وجود سایر رویدادها با تخصیص منابع به درخواست رویدادی که اکنون به سیستم رسیده نیز پاسخ دهد.
رویدادها می‌توانند با الگوی مشخص و تبعیت از یک توزیع ریاضی رخ دهند و یا آن که غیر قابل پیشبینی باشند؛ ‌الگوی فرارسیدن رویداد ها می تواند به صورت دوره ای\LTRfootnote{periodic}، تصادفی\LTRfootnote{Stochastic} و یا پراکنده\LTRfootnote{Sporadic} باشد.

در سناریو عمومی کارایی معیار سنجش پاسخ سیستم به محرک می تواند یکی از موارد زیر باشد:
\begin{itemize}
\item
تاخیر\LTRfootnote{Latency}: زمان میان فرارسیدن رویداد و پاسخ سیستم را تاخیر سیستم می‌گویند. برای مثال زمانی که یک کاربر باید منتظر پاسخ جستجوی سفر‌های مشابه در برنامه ی اشتراک خودرو بماند.
\item
مهلت های پردازشی\LTRfootnote{Deadlines in processing}:یک رویداد ممکن زمان مشخصی را برای سیستم به عنوان حداکثر زمانی که باید پاسخ رویداد آماده باشد مشخص کند،‌ تخطی از این زمان سبب کاهش کارایی سیستم خواهد شد.
\item
توان عملیاتی سیستم\LTRfootnote{The throughput of the system}:تعداد پردازه هایی سیستم در واحد زمان می تواند به پایان برساند به عنوان توان عملیاتی شناخته می شود. برای مثال، سرور یک برنامه ‌تاکسی‌برخط توان پاسخ دادن به تعداد مشخصی درخواست سفر از سمت کاربران در لحظه را دارد.
\end{itemize}
همچنین معیار های دیگری نظیر واریانس تاخیر\LTRfootnote{Jitter} و تعداد رویداد های بلوکه شده نیز وجود دارند که کارایی سیستم به وسیله ی آن ها قابل سنجش است.

در ادامه به معرفی بخش های مختلف سناریو عمومی کارایی پرداخته می‌شود.
\begin{itemize}
\item
منبع محرک\LTRfootnote{Source of stimulus} :
یک منبع از داخل و یا خارج از سیستم که بتواند رویدادی را ایجاد و به سیستم ارسال کند،‌می تواند منبع محرک سناریوی کارایی باشد.
\item
محرک\LTRfootnote{Stimulus} : 
فرا رسیدن یک رویداد
\item
محصول\LTRfootnote{Artifact}: سیستم یا بخشی از آن که مسئول رسیدگی به رویداد است.
\item
محیط\LTRfootnote{Enviroment}:
سیستم می تواند در حالت های مختلف عملیاتی مانند حالت عادی\LTRfootnote{Normal} ، اضطراری\LTRfootnote{Emergency}، اوج بار\LTRfootnote{Peak Load} یا اضافه بار\LTRfootnote{OverLoad} باشد.
\item
پاسخ\LTRfootnote{Response} :
سیستم در پاسخ رسیدن یک رویداد با تخصیص منابع به آن پاسخ می دهد که ممکن است به تغییر محیط نیز منجر شود.
\item
اندازه گیری پاسخ\LTRfootnote{Response Measure}:
معیار های سنجش کارایی سیستم پیش از این بررسی شدند.برای بررسی کارایی سیستم می توان به تاخیر،‌توان عملیاتی سیستم و مهلت های پردازشی اشاره کرد.
\end{itemize}

\section{تاکتیک ها در کارایی}
هدف تاکتیک ها در کارایی تولید پاسخ یک رویداد با توجه به محدودیت های زمانی حاکم بر کارایی است. زمان پاسخ سیستم عموما به زمان پردازش\LTRfootnote{Processing time} و یا زمان مسدود شدن \LTRfootnote{Blocking time} ختم می شود.

پردازش یک رویداد به منابع مختلفی نیاز دارد و توسط چندین مولفه\LTRfootnote{Component} سیستم، نیاز به رسیدگی دارد. مجموع زمان‌هایی که مولفه‌های مختلف سیستم با به کار‌گیری منابع به یک رویداد رسیدگی می کنند،‌ زمان پردازشی رویداد به حساب می آید. در کنار زمان پردازش، زمان مسدود شدن نیز نقش مهمی در مجموع زمان تاخیر یک رویداد دارد؛ پردازش یک رویداد ممکن است به دلیل نیاز محاسباتی بخشی از پردازش به پردازشی دیگر و یا به دلیل در دسترس نبودن یک منبع مسدود شود.برای مثال، ممکن است زیرسیستم جستجوی نزدیک ترین راننده‌ها در برنامه‌ی تاکسی‌برخط منتظر دریافت جزئیات نقشه از زیرسیستم نقشه باشد.

تاکتیک هایی که در کارایی وجود دارند به دو دسته ی زیر تقسیم می شوند:
\begin{itemize}
\item
کنترل تقاضای منابع\LTRfootnote{Control resource demand} : این دسته از تاکتیک‌ها در سمت تقاضا فعالیت می کنند و سعی می کنند تا میزان تقاضا برای منابع را کاهش دهند.
\item
مدیریت منابع\LTRfootnote{Manage resources} : این دسته از تاکتیک‌ها سعی دارند تا کار‌ها را با بازدهی بیشتری انجام دهند تا مصرف منابع از این طریق کاهش یابد.
\end{itemize}
در ادامه به بررسی بیشتر این تاکتیک ها پرداخته می شود.
\subsection{کنترل درخواست منابع}

یکی از راه های افزایش کارایی در حوزه نرم‌افزارهای حمل و نقل درون شهری، مدیریت دقیق تقاضای منابع است. این کار می تواند با کاهش تعداد رویدادهای پردازش شده با اعمال نرخ نمونه برداری یا با محدود کردن سرعت پاسخگویی سیستم به رویدادها انجام شود.
\begin{itemize}
\item
مدیریت نمونه برداری: با کاهش نرخ نمونه برداری از محیط هر چند مقداری از دقت کاسته خواهد شد اما میزان تقاضا رو می توان به مقدار خوبی کاهش داد. زمانی که جریان سازگار داده\LTRfootnote{Consistent Stream} اهمیت بیشتری داشته باشد چشم پوشی از مقداری دقت می تواند راه حل خوبی برای کاهش تقاضا به سیستم باشد.برای مثال، در زیرسیستم نقشهي برنامه ی تاکسی‌رانی برخط، ممکن است نرخ به روزرسانی محل کنونی خودرو را کاهش دهیم تا کارایی سرور در زمان هایی که بار زیادی وجود دارد، افزایش یابد.
\item
محدودیت پاسخ به رویداد ها: زمانی که رویداد‌های گسسته با نرخ بالایی به سیستم می‌رسند،‌ باید در صف منتظر شروع پردازش بمانند. به دلیل گسسته بودن این رویداد ها امکان کاهش نرخ نمونه برداری وجود ندارد، در چنین شرایطی سیستم می تواند انتخاب کند که با توجه به شرایطی نظیر اندازه‌ی صف و یا زمانی که تاکنون رویداد در پردازش به خود اختصاص داده است، ‌حداکثر زمانی که یک رویداد می تواند پردازش بگیرد را محمدود کند. با محدود کردن حداکثر زمان پردازشی که یک رویداد می تواند به خود اختصاص دهد،‌ سیستم قادر خواهد بود تا رویداد های قابل پیش‌بینی بیشتری را تضمین کند. در این روش زمانی که تصمیم گرفته می‌شود رویدادی رها\LTRfootnote{Drop} شود باید سیاستی برای مدیریت این وضعیت از پیش تعیین شود: آیا رویداد های رها شده ثبت می‌شود و یا به راحتی نادیده گرفته خواد شد؟ آیا سایر سیستم‌ها و کاربران از رها سازی رویداد مطلع خواهند شد؟
در این زمینه می توان به مواقعی اشاره کرد که سیستم زمان زیادی را برای پیدا کردن سفر‌های مشابه یک درخواستدر سیستم اشتراک خودرو صرف می کند و در چنین حالتی این تاکتیک توصیه می کند تا سرور این درخواست را رها کند تا با آزاد سازی منابع و پاسخ به درخواست های دیگر کارایی افزایش یابد.  
\item 
اولویت‌بندی رویداد‌ها: اگر منابع کافی برای پاسخ دهی به تمامی رویداد‌ها در اختیار سیستم قرار ندارد و یا اگر همه ی رویداد‌ها از اهمیت یکسانی برخوردار نیستند استفاده از یک الگوریتم اولویت بندی می‌تواند تاثیر خوبی در کارایی سیستم داشته باشد.برای مثال، در برنامه‌ی یافتن پارکینگ های درون‌شهری سیستم می تواند درخواست مشتریان را بر اساس زمان پردازشی مورد نیاز اولویت‌دهی کند.
\item
کاهش سربار: هر چند وجود واسطه‌ها در افزایش اصلاح‌پذیری\LTRfootnote{Modifiability} از اهمیت بالایی برخوردار است اما حذف این واسطه ها می تواند منجر به افزایش کارایی سیستم شود. معماری در این مساله با تقابل دو نیاز کارایی و اصلاح‌پذیری دست و پنجه نرم می کند و باید بهترین تصمیم با توجه به نیازها و اهداف سیستم اخذ شود.
برای مثال معماری می تواند بجای استفاده از چندین سیستم برای پاسخ به درخواست سفر توسط کاربر در برنامه‌ی تاکسی‌برخط از یک زیرسیستم یکپارچه استفاده کند و با حذف واسطه ها کارایی را در مجموع افزایش دهد؛هرچند همان طور که بیان شد انجام چنین کاری سبب کاهش نیاز کیفی اصلاح‌پذیری خواهد شد.
\item
زمان اجرای محدود: محدودیتی در میزان استفاده از زمان اجرا برای پاسخگویی به یک رویداد تعیین کنید. برای الگوریتم های تکراری و وابسته به داده ها ، محدود کردن تعداد تکرارها روشی برای محدود کردن زمان پردازش است که هر چند می تواند به پاسخی با دقت کمتر ختم شود اما در افزایش کارایی تاثیر مثبتی دارد.
برای مثال، در برنامه ی پارکینگ های درون شهری، معماری می‌تواند زمانی که سرور برای پیدا کردن نزدیک‌ترین مسیر تا یک پارکینگ مشخص صرف می کند را با از دست دادن مقداری قابل چشم پوشی دقت، محدود کند.
\item
افزایش بازدهی منابع: بهبود الگوریتم های مورد استفاده در مناطق بحرانی
تاخیر را کاهش می دهد.
\end{itemize}
\subsection{مدیریت منابع}
هر چند ممکن است کنترل تقاضا برای منابع همواره ممکن نباشد اما مدیریت منابع همیشه در اختیار سیستم قرار دارد. گاهی می توان برای انجام یک پردازش از منابع مختلف استفاده کرد به عنوان مثال می توان در یک سیستم داده های میانی را در حافظه‌ی نهان\LTRfootnote{Cache} نگهداری کرد و یا از حافظه های دیگر سیستم به منظور ذخیره سازی آن ها استفاده کنید. در ادامه چندین روش مدیریت منابع سیستم مطرح می شوند:
\begin{itemize}
\item
افزایش تعداد منابع: افزایش کمیت و کیفیت منابع همواره یکی از راه های کاهش تاخیر رویداد هاست که البته دغدغه های مالی را بر سر راه خود دارد.
\item 
استفاده از هم‌زمانی\LTRfootnote{Concurrency}: اگر درخواست ها قابلیت پردازش موازی را داشته باشند، ‌استفاده از هم‌زمانی و پردازش موازی همراه با الگوریتم‌هایی جهت مدیریت پردازش رویدادها بر نخ های پردازشی متفاوت به جهت حفظ توان عملیاتی و عدالت می تواند کارایی سیستم را به مقدار خوبی افزایش دهد. برای مثال درخواست‌های کاربرهای یک سرویس تاکسیرانی آنلاین تا حد خوبی از یک دیگر مجزا هستند و می‌توان به صورت همزمان انجام شود.
\item
حفظ چندین نسخه از محاسبات: وجود چندین سیستم مشابه برای پاسخ به رویداد ها، در زمانی که چندین رویداد هم‌زمان با یکدیگر رخ می دهند می تواند کارایی سیستم را تا حد زیادی افزایش دهد.برای مثال وجود چنین سرور در معماری های کلاینت-سرور\LTRfootnote{Client-Server} مانند آن‌چه که در سرویس تاکسیرانی آنلاین وجود دارد به همراه یک نرم افزار متعادل کننده ی بار\LTRfootnote{Load Balancer} می تواند به کارایی سیستم در پاسخ به رویداد ها کمک زیادی کند.
\item 
حفظ چندین نسخه از داده: تکثیر داده ها\LTRfootnote{Data Replication} شامل نگهداری نسخه های جداگانه از داده ها برای کاهش نزاع میان رویداد ها هنگام چندین دسترسی همزمان است. از آنجا که داده های تکثیر شده معمولاً یک کپی از داده های موجود است، حفظ کپی و هماهنگ سازی کپی ها به مسئولیتی تبدیل می شود که سیستم باید آن را بر عهده بگیرد. از این روش می‌توان در بیشتر نرم‌افزارهای این حوزه به خوبی استفاده کرد.
\item
محدود‌سازی اندازه ی صف: این روش که اصولا در کنار محدودیت پاسخ به رویدادها استفاده می شود،‌ اقدام به محدود سازی اندازه صف ورودی رویدادها می کند. در این روش باید سیاست مشخصی در برابر رویداد هایی که به دلیل محدودیت اندازه ی صف پذیرش نمی شوند،‌ اتخاذ شود. برای مثال در اپلیکیشن پارکینگ زمانی که پارکینگ‌ها پر است و تعداد درخواست‌ها زیاد است می‌توان از محدود سازی صف استفاده کرد.
\item
برنامه‌ریزی منابع: در این روش هدف درک ویژگی های استفاده از هر منبع و انتخاب استراتژی برنامه ریزی سازگار با آن است.
\end{itemize}
\section{طراحی فهرست بازبینی برای کارائی}
در ادامه فهرستی برای پشتیبانی از روند طراحی و تجزیه و تحلیل نیاز کارایی ارائه شده است.
\subsection{تخصیص مسئولیت ها} 
در زمان تخصیص مسئولیت‌ها باید \LTRfootnote{Allocation of Responsibilities} به موارد زیر توجه شود:
\begin{itemize}
\item
مسئولیت‌های سیستم در شرایط مختلف نظیر بارسنگین \LTRfootnote{Heavey Load} یا نیاز بحرانی به پاسخ یک رویداد تعیین شود. برای این مسئولیت ها، الزامات پردازش هر مسئولیت شناسایی شده و تعیین شود که آیا ممکن است موجب گلوگاه\LTRfootnote{Bottleneck} های مختلف شود؟
\item
مسئولیت هایی که از وجود یک رشته کنترلی میان فرآیندها و یا مرزهای پردازشی حاصل می‌شوند،‌ را شناسایی کنید.
\item
مسئولیت کنترل رشته های پردازشی در هنگام استفاده از روش های چند‌رشته ای\LTRfootnote{Multi-Threading}.
\item
اطمینان حاصل شود سیستم با وجود مسئولیت‌هایی که به عهده‌اش گذاشته شده است،‌ می تواند به نیاز های کارایی پاسخ دهد.
\end{itemize}
\subsection{مدل هماهنگی}
عناصر سیستم را که باید هماهنگ شوند - مستقیم یا غیرمستقیم - تعیین شود و رویکرد‌های ارتباطی و هماهنگی انتخاب شود که موارد زیر را انجام می دهند:
\begin{itemize}
\item
پشتیبانی از هم‌زمانی رویداد‌ها و روش‌های اولویت بندی و استراتژی های برنامه‌ریزی
\item
اطمینان حاصل شود که می‌توان پاسخ کارایی مورد نیاز را ارائه داد.
\item
سیستم در صورت لزوم می‌تواند ورودهای دوره ای، تصادفی یا پراکنده را پذیرش کند.
\item
از خصوصیات مناسب مکانیسم های ارتباطی برخوردار باشند.
\end{itemize}
\subsection{مدل داده}
مدل داده\LTRfootnote{Data Model} ای آن بخش از داده ها که قرار است بار زیادی را متحمل شوند و در کارایی سیستم اثر گذار هستند مشخص شوند و موارد زیر در مورد آنها بررسی شود.
\begin{itemize}
\item
آیا حفظ چندین نسخه از داده‌های کلیدی به نفع عملکرد است یا خیر؟
\item
آیا تقسیم\LTRfootnote{Partitioning} داده ها به نفع کارایی است؟
\item
آیا کاهش نیاز های پردازشی برای عملیات هایی که نیاز به تغییر در داده ها دارند امکان پذیر است؟
\item
آیا امکان افزایش منابع به جهت کاهش گلوگاه ها در هنگام تغییر در داده ها امکان پذیر است؟
\end{itemize}
\subsection{نقشه برداری در میان عناصر معماری}
\begin{itemize}
\item
در مواردی که بارگیری سنگین شبکه رخ می دهد، تعیین اینکه آیا قرارگیری همزمان برخی از اجزا باعث کاهش بارگزاری و بهبود کارایی کلی می شود؟
\item
اطمینان حاصل شود که مولفه ها با محاسبات سنگین تر به پردازنده هایی با بیشترین ظرفیت پردازش اختصاص داده شده اند.
\item
اطمینان حاصل شود استفاده از هم‌زمانی ممکن است و آیا تاثیر مثبت قابل توجه ای بر روی کارایی سیستم دارد؟
\item
تعیین شود که آیا انتخاب رشته های کنترل و مسئولیت های مربوط به آنها گلوگاه هایی را ایجاد می کند یا خیر؟
\end{itemize}
\subsection{مدیریت منابع}
تعیین شود کدام منابع نقش حیاتی و مهمی در کارایی را ایفا می کنند. اطمینان حاصل شود این منابع در طول اجرا تحت شرایط عادی و بار کاری بالا تحت نظارت و مدیریت صحیح قرار می گیرند.
\subsection{زمان اتصال}
موارد زیر برای مولفه هایی که پس از کامپایل \LTRfootnote{Compile} به سیستم اضافه می شوند،‌تعیین شود:
\begin{itemize}
\item
زمان لازم برای اضافه شدن مولفه به سیستم
\item
میزان سربار اضافه ناشی از اضافه شدن دیر هنگام مولفه به سیستم
\end{itemize}
باید اطمینان حاصل شود که این دو سربار جریمه خیلی زیادی بر روی سیستم نمی گذارند.
\subsection{انتخاب فناوری}
آیا انتخاب فناوری شما امکان تنظیم موارد زیر را دارد:
\begin{itemize}
\item
سیاست برنامه ریزی
\item
اولویت ها
\item
سیاست های کاهش تقاضا
\item
تخصیص بخشهایی از فناوری به پردازنده ها
\item
سایر موارد مربوط به کارایی
\end{itemize}




\section{مطالعات موردی}

در اینجا مطالعات موردی در ۳ حوزه حمل و نقل درون شهری برای نیاز کاریی انجام شده است. 
\subsection{تاکسیرانی و حمل‌و‌نقل درون شهری}

اپلیکیشن MyWay : در مستندات مربوط به این اپلیکیشن عنوان شده است که باید ۲۴ ساعته و ۷ روز هفته در دسترس باشد و از دست رفتن سرویس در هنگام شب ترجیح داده می‌شود که ۱ روز قبل‌تر از آن نیز از طریق وبسایت به اطلاع کاربران رسیده است. 
اپلیکیشن باید دارای دسترس‌پذیری و قابلیت اعتماد بالایی باشد و بازیابی فاجعه نیز به خوبی انجام شود. در شرایط میانگین سرعت اینترنت برای برنامه‌ریزی سفر درون شهری اپ باید نهایتا تا ۵ ثانیه پاسخ‌گو باشد و کاربر با استفاده از یک نوار پیشرفت باید از میزان پیشرفت مطلع باشد. 
همچنین در شرایط اینترنت عادی ارتباط اپ با سرور باید در ۲ ثانیه صورت گیرد به جز حالت برنامه‌ریزی سفر. 

اپلیکیشن OLA : دیتابیس باید به راحتی اطلاعات مربوط به حداقل ده هزار مشتری را ذخیره کند. نرم‌افزار باید قابلیت همزمانی ارتباط کاربران را پشتیبانی کند.

\subsection{اشتراک خودرو و حمل‌و‌نقل درون شهری}


\subsection{پارکنیگ های عمومی و حمل‌و‌نقل درون شهری}




