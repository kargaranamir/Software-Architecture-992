% Chapter 2
\chapter{کارایی}
\section{تعریف کارایی}
کارایی\LTRfootnote{Performance} به توانایی سیستم در پاسخ گویی به نیاز ها در زمانی مشخص اشاره دارد.زمانی که رخدادی جدید نظیر خواسته ای از جانب کاربر یا سایر سیستم ها پیش می آید سیستم باید در بازه ی زمانی مورد انتظار آن رخداد را بررسی کند و به نیاز آن پاسخ دهد.در یک سیستم مبتی بر وب ممکن است کارایی با گزاره هایی نظیر تعداد درخواست بر دقیقه بیان شود.

کارایی در بیشتر اوقات درکنار گسترش پذیری\LTRfootnote{Scalability} شنیده می شود،‌گسترش پذیری به افزایش ظریفیت پاسخ دهی با حفظ کارایی اشاره دارد و نباید مفهوم آن با کارایی یکسان در نظر گرفته شود.
\section{سناریو عمومی کارایی}
سناریو کارایی با فرارسیدن رویداد ها به سیستم آغاز می شود.سیستم باید هم زمان با وجود سایر رویداد ها با تخصیص منابع به درخواست رویدادی که اکنون به سیستم رسیده پاسخ دهد.
رویداد ها می توانند با الگوی مشخص و تبعیت از یک توزیع ریاضی رخ دهند و یا آن که غیر قابل پیشبینی باشند؛‌الگوی فرارسیدن رویداد ها می تواند به صورت دوره ای\LTRfootnote{periodic}، تصادفی\LTRfootnote{Stochastic} و یا پراکنده\LTRfootnote{Sporadic} باشد.

در سناریو عمومی کارایی معیار سنجش پاسخ سیستم به محرک می تواند یکی از موارد زیر باشد:
\begin{itemize}
\item
تاخیر\LTRfootnote{Latency}: زمان میان فرارسیدن رویداد و پاسخ سیستم را تاخیر سیستم می گویند.
\item
مهلت های پردازشی\LTRfootnote{Deadlines in processing}:یک رویداد ممکن زمان مشخصی را برای سیستم به عنوان حداکثر زمانی که باید پاسخ رویداد آماده باشد مشخص کند،‌تخطی از این زمان سبب کاهش کارایی سیستم خواهد شد.
\item
توان عملیاتی سیستم\LTRfootnote{The throughput of the system}:تعداد پردازه هایی سیستم در واحد زمان می تواند به پایان برساند به عنوان توان عملیاتی شناخته می شود.
\end{itemize}
همچنین معیار های دیگری نظیر واریانس تاخیر\LTRfootnote{Jitter} و تعداد رویداد های بلوکه شده نیز وجود دارند که کارایی سیستم به وسیله ی آن ها قابل سنجش است.

در ادامه به معرفی بخش های مختلف سناریو عمومی کاریی می پردازیم.
\begin{itemize}
\item
منبع محرک\LTRfootnote{Source of stimulus} :
یک منبع از داخل و یا خارج از سیستم که بتواند رویدادی را ایجاد و به سیستم ارسال کند،‌می تواند منبع محرک سناریوی کارایی باشد.
\item
محرک\LTRfootnote{Stimulus} : 
فرارسیدن یک رویداد
\item
محصول\LTRfootnote{Artifact}: سیستم یا بخشی از آن که مسئول رسیدگی به رویداد است.
\item
محیط\LTRfootnote{Enviroment}:
سیستم می تواند در حالت های مختلف عملیاتی مانند حالت عادی\LTRfootnote{Normal} ، اضطراری\LTRfootnote{Emergency} ، اوج بار\LTRfootnote{Peak Load} یا اضافه بار\LTRfootnote{OverLoad} باشد.
\item
پاسخ\LTRfootnote{Response} :
سیستم در پاسخ رسیدن یک رویداد با تخصیص منابع به آن پاسخ می دهد که ممکن است به تغییر محیط نیز منجر شود.
\item
اندازه گیری پاسخ\LTRfootnote{Response Measure}:
معیار های سنجش کارایی سیستم پیش از این بررسی شدند.برای بررسی کارایی سیستم می توان به تاخیر،‌توان عملیاتی سیستم و مهلت های پردازشی اشاره کرد.
\end{itemize}

\section{تاکتیک ها در کارایی}
هدف تاکتیک ها در کارایی تولید پاسخ یک رویداد با توجه به محدودیت های زمانی حاکم بر کارایی است.زمان پاسخ سیستم عموما به زمان پردازش\LTRfootnote{Processing time} و یا زمان مسدود شدن \LTRfootnote{Blocking time} ختم می شود.
\section{طراحی فهرست بازبینی برای کارائی}

در ادامه سعی می کنیم تا فهرستی را برای پشتیبانی از روند طراحی و تجزیه و تحلیل نیاز کارایی ارائه کنیم.


\subsection{تخصیص مسئولیت ها} 
در زمان تخصیص مسئولیت ها باید \LTRfootnote{Allocation of Responsibilities} به موارد زیر توجه شود:
\begin{itemize}
\item
معمار سیستم باید مشخص کند که انجام کدام یک از وظائف سیستم نیازمند همکاری با سایر سیستم هاست.
\item
اطمینان حاصل کنید مسئولیت هایی به جهت تشخیص درخواست های همکاری با سیستم های خارجی شناخته شده و یا ناشناخته اختصاص یافته اند.
\item
معماری باید مسئولیت پاسخ به وظائف زیر را در سیستم لحاظ کرده باشد:
\begin{itemize}
\item
پذیرش درخواست همکاری
\item
تبادل اطلاعات
\item
عدم پذیرش درخواست همکاری
\item
اعلان و آگاهی سازی هویت های مرتبط با سیستم از همکاری با سایر سیستم ها
\item
ثبت درخواست
\end{itemize}
\end{itemize}
\subsection{مدل هماهنگی}
مدل هماهنگی\LTRfootnote{Coordination Model} باید با نظر نظر گرفته دغدغه و نیاز های معماری به آن ها پاسخ دهد.مواردی که برای پاسخ دهی به نیاز کارایی\LTRfootnote{Performance} باید توسط مدل هماهنگی مورد توجه قرار گیرند شامل موارد زیر است:
\begin{itemize}
\item
حجم ترافیکی که به صورت مستقیم توسط سیستم های تحت کنترل و حتی سیستم هایی که شما کنترلی بر روی آن ها ندارید ایجاد می شود.
\item
ارسال به موقع پیام ها از سمت سیستم شما
\item
هم زمانی\LTRfootnote{Currency} ارسال پیام از سمت سیستم شما
\item
و شاید مهم تر از همه این است که اطمینان حاصل کنید سیستم های تحت کنترل شما فرضیات قابل انطابقی در همکاری با سایر سیستم هایی که شما کنترلی بر روی آن ها ندارید،‌در نظر گرفته اند.
\end{itemize}
\subsection{مدل داده}
مدل داده\LTRfootnote{Data Model}  از مهم ترین بخش هایی است که اگر به درستی بر آن نظارت صورت نپذیرد می تواند عواقب سنگینی را به همراه داشته باشد.در مدل داده ای باید از برداشت یکسان طرفین همکاری از اطلاعات اطمینان حاصل کنید.به این منظورم لازم است تا انتزاعات اصلی داده های مبادله شده از نظر نحو و معناشناسی مورد بررسی دقیق قرار بگیرند.
\subsection{نقشه برداری در میان عناصر معماری}
جدا از ملاحظاتی که در نقشه برداری در میان عناصر معماری\LTRfootnote{Mapping among Architectural Elements} درباره امنیت،‌ دسترس‌پذیری و کارائی وجود دارد و در سایر فصل های این مستند به آن ها پرداخته خواهد شد؛ در رابطه با قابلیت همکاری مساله ی مهم نگاشت درست از اعضای سازنده ی پردازنده ها است.
\subsection{مدیریت منابع}
لازم است معمار نرم افزار اطمینان حاصل کند منابعی که سیستم به جهت همکاری با سایر سیستم ها نیاز دارد هیچ گاه سیستم را تحت فشار غیر قابل تحمل قرار نخواهد داد و بار منابع تحمیل شده توسط الزامات همکاری همواره قابل قبول است.

در ضمن نیاز یک سیستم ناظر با هدف تخصیص منابع به صورت منصفانه و بر پایه ی سیاست های تبیین شده در معماری باید حتما دیده شود.
\subsection{زمان اتصال}
لازم است سیستم هایی که ممکن است با یکدیگر همکاری داشته باشند شناسایی شده و از رعایت موارد زیر در مورد آن ها اطمینان حاصل شود.
\begin{itemize}
\item
از وجود یک سیاست مشخص به منظور همکاری با سیستم های شناخته شده و ناشناخته خارجی اطمینان حاصل کنید.
\item
از وجود قوانینی برای عدم پذیرش درخواست ها و ثبت درخواست هایی که پذیرفته نشده اند اطمینان حاصل کنید.
\item
در صورت تاخیر در اتصال ، اطمینان حاصل کنید که مکانیزم هایی از کشف خدمات یا پروتکل های جدید مرتبط یا ارسال اطلاعات با استفاده از پروتکل های انتخاب شده پشتیبانی می کنند.
\end{itemize}
\subsection{انتخاب فناوری}
در انتخاب فناوری باید روی این مساله تمرکز نمائید که انتخاب هر فناوری چه تاثیراتی بر روی رابط ها و همکاری سیستم با دیگر سیستم ها خواهد داشت.بررسی کنید آیا فناوری آیای انتخاب شده توانایی پاسخ گویی به نیاز هایی که توسط قابلیت همکاری مطرح می شوند را دارا هستند و یا خیر؟








